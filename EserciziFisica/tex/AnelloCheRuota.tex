\documentclass[../main.tex]{subfiles} 
\begin{document}

\exercise{Anello che ruota} %acr
\textex
Un anello ruota attorno ad un suo diametro (parallelo alla direzione della forza di gravità) ad una velocità
angolare $\omega$ costante nel tempo. Su di esso vi è un punto materiale di massa $m$, vincolato a muoversi lungo l'anello e su cui agisce
la forza di gravità.

Determinare le posizioni di equilibrio stabile ed instabile in funzione di $\omega$ del punto materiale e studiare la
frequenza delle piccoli oscillazioni attorno ai punti di equilibrio stabile.
\solution
Mi pongo nel sistema di riferimento non inerziale che si muove solidalmente all'anello. Grazie all'equazione 
(AGGIUNGERE ELENCO FORMULE)
ho che:
\begin{equation*}
	\overrightarrow{a}=\overrightarrow{a_{O'}}+\overrightarrow{a_r}+2\overrightarrow{\omega}\times \overrightarrow{v_r} 
	+ \dot{\overrightarrow{\omega}}\times \overrightarrow{r}+\overrightarrow{\omega}\times(\overrightarrow{\omega}
	\times\overrightarrow{r})
\end{equation*}
In questo caso particolare però, $O'$ (il centro del sistema di riferimento non inerziale) coincide con il centro
dell'anello, quindi $\overrightarrow{a_{O'}}=0$; inoltre $\overrightarrow{\omega}=\omega \hat{z}$, perciò
$\dot{\overrightarrow{\omega}}=0$. Mi riconduco quindi a:
\begin{equation*}
	\overrightarrow{a}=\overrightarrow{a_r}+2\omega\hat{z}\times \overrightarrow{v_r} 
	+\omega\hat{z}\times(\omega\hat{z}\times\overrightarrow{r})
\end{equation*}
Voglio trovare ora la componente di $\overrightarrow{a}$ diretta lungo $\hat{\theta}$, considero quindi ogni singolo
addendo:
\begin{itemize}
	\item	$\overrightarrow{a_r}=(\ddot{r}-r\dot{\theta}^2)\hat{r}+(2\dot{r}\dot{\theta}+r\ddot{\theta})\hat{\theta}$
			per l'equazione..., quindi la sua componente diretta lungo $\theta$ è uguale a $
			2\dot{r}\dot{\theta}+r\ddot{\theta}=r\ddot{\theta}$, in quanto $r$ è costante.
	\item	$\hat{z}$ e $v_r$ giacciono sul piano contenente l'anello, quindi $\omega\hat{z}\times v_r$ ne è perpendicolare
			e perciò non ha componente lungo $\hat{\theta}$.
	\item	Infine $\omega\hat{z}\times(\omega\hat{z}\times\overrightarrow{r})=
			(-r\omega^2\sin\theta,0,0)=-r\omega^2\sin\theta^2\hat{r}-r\omega^2\sin\theta\cos\theta\hat{\theta}$,
			quindi il valore della componente diretta lungo $\hat\theta$ è $-r\omega^2\sin\theta\cos\theta$.
\end{itemize}
Ottengo quindi che $a_\theta=r\ddot{\theta}-r\omega^2\sin\theta\cos\theta$.
Inoltre so che $F_\theta=m a_\theta$ e che $F_\theta=-g\sin\theta$, da cui:
\begin{equation}\label{acr:EquazioneDifferenziale}
	r\ddot{\theta}=r\omega^2\sin\theta\cos\theta-g\sin\theta
\end{equation}
Le posizioni di equilibrio si hanno quando $\ddot\theta=0$ in \cref{acr:EquazioneDifferenziale}, cioè
$\sin\theta=0$ (cioè $\theta=0$ o $\theta=\pi$) o $\cos\theta=\frac{g}{\omega^2r}$. Distinguiamo quindi tre casi:
\begin{itemize}
	\item $\theta=0$.
\end{itemize}



\solution[2]

\end{document}