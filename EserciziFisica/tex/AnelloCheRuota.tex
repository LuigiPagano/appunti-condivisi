\documentclass[../main.tex]{subfiles} 
\begin{document}

\exercise{Anello che ruota} %acr
\textex
Un anello ruota attorno ad un suo diametro (parallelo alla direzione della forza di gravità) ad una velocità
angolare $\omega$ costante nel tempo. Su di esso vi è un punto materiale di massa $m$, vincolato a muoversi lungo l'anello e su cui agisce
la forza di gravità.

Determinare le posizioni di equilibrio stabile ed instabile in funzione di $\omega$ del punto materiale e studiare la
frequenza delle piccoli oscillazioni attorno ai punti di equilibrio stabile.
\solution
Mi pongo nel sistema di riferimento non inerziale che si muove solidalmente all'anello. Grazie all'equazione 
\cref{AccNonInerziale}
ho che:
\begin{equation*}
	\overrightarrow{a}=\overrightarrow{a_{O'}}+\overrightarrow{a_r}+2\overrightarrow{\omega}\times \overrightarrow{v_r} 
	+ \dot{\overrightarrow{\omega}}\times \overrightarrow{r}+\overrightarrow{\omega}\times(\overrightarrow{\omega}
	\times\overrightarrow{r})
\end{equation*}
In questo caso particolare però, $O'$ (il centro del sistema di riferimento non inerziale) coincide con il centro
dell'anello, quindi $\overrightarrow{a_{O'}}=0$; inoltre $\overrightarrow{\omega}=\omega \hat{z}$, perciò
$\dot{\overrightarrow{\omega}}=0$. Mi riconduco quindi a:
\begin{equation*}
	\overrightarrow{a}=\overrightarrow{a_r}+2\omega\hat{z}\times \overrightarrow{v_r} 
	+\omega\hat{z}\times(\omega\hat{z}\times\overrightarrow{r})
\end{equation*}
Voglio trovare ora la componente di $\overrightarrow{a}$ diretta lungo $\hat{\theta}$, considero quindi ogni singolo
addendo:
\begin{itemize}
	\item	$\overrightarrow{a_r}=(\ddot{r}-r\dot{\theta}^2)\hat{r}+(2\dot{r}\dot{\theta}+r\ddot{\theta})\hat{\theta}$
			per l'equazione..., quindi la sua componente diretta lungo $\theta$ è uguale a $
			2\dot{r}\dot{\theta}+r\ddot{\theta}=r\ddot{\theta}$, in quanto $r$ è costante.
	\item	$\hat{z}$ e $v_r$ giacciono sul piano contenente l'anello, quindi $\omega\hat{z}\times v_r$ ne è perpendicolare
			e perciò non ha componente lungo $\hat{\theta}$.
	\item	Infine $\omega\hat{z}\times(\omega\hat{z}\times\overrightarrow{r})=
			(-r\omega^2\sin\theta,0,0)=-r\omega^2\sin\theta^2\hat{r}-r\omega^2\sin\theta\cos\theta\hat{\theta}$,
			quindi il valore della componente diretta lungo $\hat\theta$ è $-r\omega^2\sin\theta\cos\theta$.
\end{itemize}
Ottengo quindi che $a_\theta=r\ddot{\theta}-r\omega^2\sin\theta\cos\theta$.
Inoltre so che $F_\theta=m a_\theta$ e che $F_\theta=-g\sin\theta$, da cui:
\begin{equation}\label{acr:EquazioneDifferenziale}
	r\ddot{\theta}=r\omega^2\sin\theta\cos\theta-g\sin\theta
\end{equation}
Le posizioni di equilibrio si hanno quando $\ddot\theta=0$ in \cref{acr:EquazioneDifferenziale}, cioè
$\sin\theta=0$ (cioè $\theta=0$ o $\theta=\pi$) o $\cos\theta=\frac{g}{\omega^2r}$. 

Consideriamo quindi ora un angolo $\theta$ molto vicino ad un angolo $\theta_0$ di equilibrio e sviluppiamo il seno ed 
il coseno rispetto a quel punto:
\begin{equation}\label{acr:SviluppoSeno}
	\sin\theta=\sin\theta_0+\cos\theta_0(\theta-\theta_0)+\mathcal{O}((\theta-\theta_0)^2)
\end{equation}
\begin{equation}\label{acr:SviluppoCoseno}
	\cos\theta=\cos\theta_0-\sin\theta_0(\theta-\theta_0)+\mathcal{O}((\theta-\theta_0)^2)
\end{equation}
Sostituendo nella \cref{acr:EquazioneDifferenziale} e utilizzando che $r\omega^2\sin\theta_0\cos\theta_0-g\sin\theta_0$, 
otteniamo:
\begin{equation*}
\begin{split}
	r\ddot{\theta}	=& r\omega^2\left[\sin\theta_0\cos\theta_0+\left(\cos^2\theta_0-\sin^2\theta_0\right)
					(\theta-\theta_0)+\mathcal{O}((\theta-\theta_0)^2)\right]\\
					& -g\left(\sin\theta_0+\cos\theta_0(\theta-\theta_0)+\mathcal{O}((\theta-\theta_0)^2)\right)\\
					=& \left[ r\omega^2\left(2\cos^2\theta_0-1\right)-g\cos\theta_0 \right](\theta-\theta_0)
					+\mathcal{O}((\theta-\theta_0)^2)
\end{split}
\end{equation*}
Quindi $\theta_0$ è un punto di equilibrio stabile se e solo se :
\begin{equation}\label{}
	r\omega^2\left(2\cos^2\theta_0-1\right)-g\cos\theta_0<0
\end{equation}
cioè se vicino a $\theta_0$ il moto si può approssimare ad un moto armonico.

Distinguiamo ora tre casi:
\begin{itemize}
	\item 	Se $\theta_0=0$, allora $r\omega^2\left(2\cos^2\theta_0-1\right)-g\cos\theta_0=r\omega^2-g$,
			che è minore di 0 se e solo se $r\omega^2<g$. In particolare in tal caso il periodo delle piccole
			oscillazioni attorno a $\theta_0=0$ è:
			\begin{equation*}
				T=\frac{2\pi}{\sqrt{g-r\omega^2}}
			\end{equation*}
	\item	Se invece $\theta_0=\pi$, vale che 
			$r\omega^2\left(2\cos^2\theta_0-1\right)-g\cos\theta_0=r\omega^2+g$ e quindi $\pi$
			non è mai un punto di equilibrio stabile.
	\item	Infine se $\cos\theta_0=\frac{g}{r\omega^2}$, si ha che
			$r\omega^2\left(2\cos^2\theta_0-1\right)-g\cos\theta_0=\frac{g^2}{r\omega^2}-r\omega^2$, che
			è minore di 0 se è solo se $g<r\omega^2$ e in quel quaso il periodo delle piccole oscillazioni
			attorno a $\theta_0$ è:
			\begin{equation*}
				T=\frac{2\pi}{\sqrt{r\omega^2-g^2/(r\omega^2)}}
			\end{equation*}
\end{itemize}

Nella soluzione non sono stati trattati i casi in cui $g=r\omega^2$ 
(sia con $\theta_0=0$ che con $\cos\theta_0=\frac{g}{r\omega^2}$) che richiedono semplicemente
sviluppi del seno e del coseno ad ordini via via successivi.


\solution[2]
Si può risolvere il problema anche rimanendo in un sistema di riferimento inerziale, in particolare vedendo
il moto come un moto unidimensionale.

L'energia $E$ è una costante del moto, poichè agiscono solo forse conservative, e quindi posso scrivere:
\begin{equation*}
	E=\frac{1}{2}m\overrightarrow{v}^2-mgr\cos\theta
\end{equation*}
Da cui, utilizzando l'equazione \cref{VelCooSferiche}, ottengo:
\begin{equation*}
	E=\frac{1}{2}mr^2\dot{\theta}^2+\frac{1}{2}mr^2\omega^2\sin^2\theta-mgr\cos\theta
\end{equation*}
Che può essere studiato come un moto unidimensionale con variabile $\theta$, in cui il potenziale
efficace è:
\begin{equation}\label{acr:PotenzialeEfficace}
	\frac{1}{2}mr^2\omega^2\sin^2\theta-mgr\cos\theta
\end{equation}

Un punto di equilibrio stabile si ha quando il potenziale efficace è in un punto di minimo locale, cioè
quando la derivata prima è uguale a 0 e la derivata seconda è positiva; in formule derivando la
\cref{acr:PotenzialeEfficace}:
\begin{equation}
	mr\left( r\omega^2\sin\theta\cos\theta+g\sin\theta \right)=0
\end{equation}
\begin{equation}
	mr\left( 2r\omega^2\cos^2\theta-r\omega^2+g\cos\theta \right)>0
\end{equation}





\end{document}