\section{Funzione Beta}

\begin{definition}[Funzione Beta]\label{FunzioneBeta} 
	La funzione Beta è definita da $\mathbb{R^+}^2$ in $\mathbb{R^+}$ come:
	\begin{equation*}
		\mathrm{B} (x,y)=\int_0^1 t^{x-1}(1-t)^{y-1} \mathrm{d}t
	\end{equation*}
\end{definition}

\begin{lemma}\label{BetaSimmetrica} 
	La funzione Beta è simmetrica, in formule:
	\begin{equation*}
		\mathrm{B}(x,y)=\mathrm{B}(y,x)
	\end{equation*}
\end{lemma}
\begin{proof}
	Basta applicare la sostituzione $t'=1-t$ nell'integrale della definizione \cref{FunzioneBeta} 
	per ottenere esattamente la simmetria della funzione Beta.
\end{proof}

\begin{lemma}\label{BetaTrigonometrica} 
	La funzione Beta rispetta la seguente identità per ogni $x,y\in \mathbb{R^+}$:
	\begin{equation*}
		\mathrm{B} (x,y)=2\int_0^{\frac\pi2} \sin(u)^{2x-1}\cos(u)^{2y-1}\mathrm{d}u
	\end{equation*}
\end{lemma}
\begin{proof}
	Trasformo l'integrale della \cref{FunzioneBeta} con la sostituzione $t=\sin^2u$.\\
	Gli estremi d'integrazione diventano $0,\frac{\pi}2$ e risulta che $\mathrm{d}t=2\sin(u)\cos(u)du$. Unendo questi risultati ottengo:
	\begin{equation*}\begin{split}
		\mathrm{B}(x,y) & =\int_0^1 t^{x-1}(1-t)^{y-1} \mathrm{d}t = \int_0^{\frac\pi2} \sin(u)^{2(x-1)}\cos(u)^{2(y-1)}2\sin(u)\cos(u)\mathrm{d}u\\
		& = 2\int_0^{\frac\pi2} \sin(u)^{2x-1}\cos(u)^{2y-1}\mathrm{d}u
	\end{split}\end{equation*}
\end{proof}

\begin{lemma}\label{FunzionaleBeta1}
	Per ogni $x,y\in\mathbb{R^+}$ la Beta rispetta la seguente equazione funzionale:
	\begin{equation*}
		\mathrm{B}(x+1,y)+\mathrm{B}(x,y+1)=\mathrm{B}(x,y)
	\end{equation*}
\end{lemma}
\begin{proof}
	Basta applicare la definizione \cref{FunzioneBeta} ottenendo:
	\begin{equation*}
		\mathrm{B}(x+1,y)+\mathrm{B}(x,y+1)=\int_0^1 t^x(1-t)^{y-1}+t^{x-1}(1-t)^y\mathrm{d}t=
		\int_0^1 t^{x-1}(1-t)^{y-1}(t+(1-t))\mathrm{d}t=\mathrm{B}(x,y)
	\end{equation*}

\end{proof}

\begin{lemma}\label{FunzionaleBeta2}
	Per ogni $x,y\in\mathbb{R^+}$ la Beta rispetta anche l'equazione funzionale:
	\begin{equation*}
		y\cdot\mathrm{B}(x+1,y)=x\cdot\mathrm{B}(x,y+1)
	\end{equation*}
\end{lemma}
\begin{proof}
	Integrando per parti vale la seguente identità:
	\begin{equation}\label{QuasiFunzionaleBeta2}
		\int_0^1 t^{x}(1-t)^{y-1}\mathrm{d}t=\left[\frac{-t^x(1-t)^y}y\right]_0^1-\int_0^1\frac{-xt^{x-1}(1-t)^y}{y}\mathrm{d}t=
		\frac xy \int_0^1 t^{x-1}(1-t)^y\mathrm{d}t 
	\end{equation}
	Sostituendo la \cref{FunzioneBeta} nella \cref{QuasiFunzionaleBeta2} si ottiene:
	\begin{equation*}
		\mathrm{B}(x+1,y)=\frac xy \cdot \mathrm{B}(x,y+1)
	\end{equation*}
	che è equivalente alla tesi.
\end{proof}


\begin{corollary}\label{FunzionaleBeta3}
	Per ogni $x,y\in\mathbb{R^+}$ la Beta rispetta:
	\begin{equation*}
		B(x+1,y)=\frac{x}{x+y}\cdot B(x,y)
	\end{equation*}
\end{corollary}
\begin{proof}
	\Cref{FunzionaleBeta1,FunzionaleBeta2} implicano:
	\begin{equation}
		\left\{
		\begin{aligned}
			&\mathrm{B}(x+1,y) &+& &\mathrm{B}(x,y+1)  & =\mathrm{B}(x,y)\\
			y\cdot &\mathrm{B}(x+1,y) &-& x&\mathrm{B}(x,y+1) & =0
		\end{aligned}
		\right.
	\end{equation}
	Che è un sistema lineare nelle variabili $\mathrm{B}(x+1,y), \mathrm{B}(x,y+1)$ se si considera $\mathrm{B}(x,y)$ costante.\\
	Risolvendo nella variabile $\mathrm{B}(x+1,y)$ si ottiene esattamente la tesi del corollario.
\end{proof}

\begin{lemma}\label{BetaLogConvessa} 
	La funzione Beta è log-convessa in entrambi gli argomenti.
\end{lemma}
\begin{proof}
	Basta dimostrarlo per un argomento (considerando l'altro fissato) e poi grazie alla simmetria mostrata in \cref{BetaSimmetrica}
	si ottiene la tesi anche per l'altro.
	
	Resta quindi da dimostrare che per ogni $a,b,y\in\mathbb{R^+}$ e $\lambda,\mu\in\mathbb{R^+}$ che rispettano $\lambda+\mu=1$ vale la seguente:
	\begin{equation}\begin{split}\label{QuasiBetaLogConvessa}
		\log \mathrm{B}(\lambda a+\mu b, y )  & \le \lambda \log \mathrm{B}(a, y ) + \mu\log \mathrm{B}( b, y )\\
		& \Updownarrow  \\
		\mathrm{B}(\lambda a+\mu b, y ) & \le  \mathrm{B}(a,y)^{\lambda}\mathrm{B}(b, y )^{\mu}
	\end{split}\end{equation}
	Ora sostituisco nella \cref{QuasiBetaLogConvessa} la definizione \cref{FunzioneBeta} ottenendo che devo dimostrare:
	\begin{equation*}
		\int_{0}^1 t^{\lambda a+\mu b}(1-t)^y\mathrm{d}y=\int_{0}^1 \left(t^a(1-t)^y\right)^{\lambda}\left(t^b(1-t)^y\right)^{\mu}\mathrm{d}t \le
		\left(\int_{0}^1 t^a(1-t)^y\right)^{\lambda}\left(\int_{0}^1 t^b(1-t)^y\right)^{\mu}
	\end{equation*}
	e questa è vera per la disuguaglianza di Holder in forma integrale.

\end{proof}

\begin{theorem}\label{GammaBeta}
	Vale la seguente relazione tra funzione Gamma e Beta (con $x,y\in\mathbb{R}^+$):
	\begin{equation*}
		\mathrm{B}(x,y)=\frac{\Gamma(x)\Gamma(y)}{\Gamma(x+y)}
	\end{equation*}
\end{theorem}
\begin{proof}
	Fissato $y$ reale positivo, sia $f_y:\mathbb{R^+}\to\mathbb{R^+}$ la funzione che rispetta:
	\begin{equation}\label{QuasiGamma}
		f_y(x)=\frac{\mathrm{B}(x,y)\Gamma(x+y)}{\Gamma(y)}
	\end{equation}
	
	Dimostro che $f_y$ rispetta le ipotesi di \cref{BohrMollerup}.
	\begin{itemize}
		\item $f_y(1)=1$
		
			Vale, sfruttando \cref{QuasiGamma,FunzionaleGamma} che:
			\begin{equation}\label{QuasiGammaDiUno}
				f_y(1)=\frac{\mathrm{B}(1,y)\Gamma(1+y)}{\Gamma(y)}=\mathrm{B}(1,y)y
			\end{equation}
			
			Però dalla definizione \cref{FunzioneBeta} e dalla simmetria \cref{BetaSimmetrica} ho anche:
			\begin{equation}\label{BetaDiUno}
				\mathrm{B}(1,y)=\mathrm{B}(y,1)=\int_0^1 t^{y-1} \mathrm{d}t = \left[\frac{t^y}y\right]_0^1= \frac1y
			\end{equation}
			
			Sostituendo \cref{BetaDiUno} nella \cref{QuasiGammaDiUno} ottengo $f_y(1)=1$.
		\item $f_y$ è log-convessa
		
			Applicando la log convessità di Gamma e Beta (\cref{BetaLogConvessa,GammaLogConvessa}) 
			nella \cref{QuasiGamma} ho che $f_y$ è prodotto di funzioni log-convesse, perciò è essa stessa log-convessa.
		\item $f_y(x+1)=xf_y(x)$
		
			Applicando l'equazione funzionale della Gamma nella \cref{QuasiGamma} si ha facilmente:
			\begin{equation}\label{QuasiFunzionaleBeta}
				f_y(x+1)=\frac{\mathrm{B}(x+1,y)(x+y)\Gamma(x+y)}{\Gamma(y)}=(x+y) f_y(x) \frac{B(x+1,y)}{\mathrm{B}(x,y)}
			\end{equation}
			Ed ora applico \cref{FunzionaleBeta3} nella \cref{QuasiFunzionaleBeta} ottenendo quando desiderato:
			\begin{equation*}
				f_y(x+1)=(x+y) f_y(x) \frac{x}{x+y}=xf_y(x)
			\end{equation*}
	\end{itemize}
	Poichè $f_y$ rispetta tutte le ipotesi di \cref{BohrMollerup}, lo applico ottenendo che $f_y=\Gamma$.
	
	Di conseguenza, ricordando la definizione \cref{QuasiGamma} ottengo:
	\begin{equation*}
		\frac{\mathrm{B}(x,y)\Gamma(x+y)}{\Gamma(y)}=f_y(x)=\Gamma(x) \Longrightarrow \mathrm{B}(x,y)=\frac{\Gamma(x)\Gamma(y)}{\Gamma(x+y)}
	\end{equation*}
\end{proof}

\begin{remark}[Valore di $\Gamma\left(\frac12\right)$]
	Ponendo $x=y=1$ in \cref{GammaBeta} e sfruttando \cref{BetaTrigonometrica} ottengo:
	\begin{equation*}
		\mathrm{B}\left(\frac12,\frac12\right)=\dfrac{\Gamma\left(\frac12\right)^2}{\Gamma(1)}\Rightarrow 
		\Gamma\left(\frac12\right)^2=2\int_0^{\frac{\pi}2}\sin^0u\cos^0u\mathrm{d}u=\pi\Rightarrow \Gamma\left(\frac12\right)=\sqrt{\pi}
	\end{equation*}
\end{remark}