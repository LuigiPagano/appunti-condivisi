\section{Differenziazione della gamma}\label{dg}
Questa sezione è la prima a dover usare esplicitamente ragionamenti del tipo \emph{epsilon, delta}.
Per evitare di usare questi ragionamenti si dovrebbero applicare teoremi forti riguardo la possibilità
di scambiare tra loro gli operatori di integrale, di derivata e di limite.
Tuttavia questi teoremi prescindono dal programma di analisi I e perciò abbiamo deciso di trovare strade
che li evitino, mantenendo le dimostrazioni le più elementari possibili.

Dimostreremo che la Gamma è una funzione derivabile infinite volte ed espliciteremo le sue derivate.
Inoltre studieremo alcune proprieta della digamma (derivata logartmica della gamma).

\begin{definition}
	Sia $f_h:\mathbb{R}\to\mathbb{R}$, con $h\in\mathbb{R^+}$, la funzione definita come:
	\begin{equation*}
		f_h(t)=\frac{t^h-1}h
	\end{equation*}

\end{definition}

\begin{lemma}\label{dg:LagrangeApprox}
	Fissati $t,h\in\mathbb{R^+}$, esiste $0\le h'\le h$ tale che:
	\begin{equation*}
		f_h(t)=\log{t}\cdot t^{h'}
	\end{equation*}
\end{lemma}
\begin{proof}
	Sia $f:\mathbb{R}\to\mathbb{R}$ la funzione definita come $g(x)=t^x$.
	
	Applicando le regole standard di derivazione si ha $g'(x)=\log{t}\cdot t^x$.
	
	Applico il teorema di lagrange con estremi $[0,h]$ alla funzione $g$ ottenendo la tesi:
	\begin{equation*}
		\exists 0\le h'\le h\ :\ f_h(t)=\frac{g(h)-g(0)}{h-0}=g'(h')=\log{t}\cdot t^{h'}
	\end{equation*}
\end{proof}



\begin{lemma}\label{dg:DisEstremale}
	Fissati $t,h\in\mathbb{R^+}$ con $h\le1$ risulta vera la disuguaglianza:
	\begin{equation*}
		\left\lvert f_h(t)-\log{t}\right\rvert\le \left\lvert\log{t}\right\rvert\max(1,t)
	\end{equation*}
\end{lemma}
\begin{proof}
	Applicando \cref{dg:LagrangeApprox} ottengo che vale la catena di identità (con $0<h'<h$):
	\begin{equation}\label{dg:FurbaId}
		\left\lvert f_h(t)-\log{t}\right\rvert=\left\lvert\log{t}\right\rvert \cdot \left\lvert t^{h'}-1\right\rvert
	\end{equation}
	
	Per $t\ge 1$ ho che vale (sfruttando $h'\le h\le 1$ $0\le t^{h'}-1\le t-1<t$.\\
	Per $0\le t<1$ ho che vale (sfruttando $h'\ge0$) $-1\le t^{h'}-1\le 0$.\\
	Unendo questi due risultati ottengo facilmente:
	\begin{equation}\label{dg:StupidaDis}
		\left\lvert t^{h'}-1\right\rvert \le \max(1,t)
	\end{equation}
	Applicando \cref{dg:StupidaDis} in \cref{dg:FurbaId} ottengo la tesi del lemma.
\end{proof}

\begin{lemma}
	Fissato un intervallo $[a,b]$ con $0<a<b$, per $h\to 0$ le funzioni $f_h$ convergono uniformemente alla funzione $\log$.
\end{lemma}
\begin{proof}
	Questa è una facile conseguenza di \cref{dg:LagrangeApprox}.
\end{proof}

\begin{theorem}\label{dg:GammaDerivata}
	La funzione Gamma è derivabile e la derivata rispetta:
	\begin{equation}
		\Gamma'(x)=\int_0^{\infty} \log{t}\cdot t^{x-1}e^{-t}\mathrm{d}t
	\end{equation}
\end{theorem}


