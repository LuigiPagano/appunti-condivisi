\section{Approssimazioni della funzione Gamma}
\begin{lemma}
Siano $f_n$, con $n\in\mathbb{N}$, e $g$ funzioni definite in $(0,\infty)$ e Riemann-integrabili su $[a,b]$ per ogni
$0<a<b<\infty$. Se valgono le seguenti proprietà:
\begin{itemize}
 \item $|f_n|\le g$ per ogni $n\in\mathbb{N}$;
 \item $f_n\to f$ uniformemente in ogni sottoinsieme compatto di $(0,\infty)$;
 \item $\int_0^\infty{g(x)dx}<\infty$.
\end{itemize}
Allora vale che:
\begin{equation*}
 \lim_{x\to\infty}\int_0^\infty f_n(x)dx=\int_0^\infty f(x)dx
\end{equation*}
\end{lemma}

\begin{proof}
 Dimostro innanzitutto che per ogni $0<a<b<\infty$, ho che $f$ è integrabile su $[a,b]$ e in particolare vale:
 \begin{equation*}
  \int_a^b{f(x)dx}=\lim_{n->\infty}\int_a^b{f_n(x)dx}
 \end{equation*}
  Sia $\sigma_m$ la suddivisione equispaziata dell'intervallo $[a,b]$ di nodi $x_i=a+\frac{i}{m}(b-a)$,
  allora $f$ è Riemann-integrabile su $[a,b]$ se per ogni $\varepsilon>0$ esiste $M$ tale che se $m\ge M$
  allora $|S(f,\sigma_m)-s(f,\sigma_m)|<\varepsilon$.
 Per la disuguaglianza triangolare vale però che:
 \begin{equation*}
  |S(f,\sigma_m)-s(f,\sigma_m)|\le  
  |S(f,\sigma_m)-S(f_n,\sigma_m)|+ |s(f_n,\sigma_m)-s(f,\sigma_m)|+ |S(f_n,\sigma_m)-s(f_n,\sigma_m)|
 \end{equation*}
 Dato che $f_n\to f$ uniformemente su $[a,b]$, allora per ogni $\mu>0$ esiste $N$ tale che per ogni
 $n\ge N$ e per ogni $x\in[a,b]$ vale $|f(x)-f_n(x)|<\mu$, quindi vale facilmente che per ogni $m\ge M$:
 \begin{equation} \label{ViciniInIntervallo}
 \begin{split}
 |S(f,\sigma_m)-S(f_n,\sigma_m)| & <\mu(b-a) \\
 |s(f,\sigma_m)-s(f_n,\sigma_m)| & <\mu(b-a)
 \end{split}
 \end{equation} 
 E dato che $f_n$ è Riemann-integrabile
 su $[a,b]$ per ogni $n$, allora per ogni $\mu>0$ esiste $M$ tale che per ogni $m\ge M$ vale:
 \begin{equation}
  |S(f_n,\sigma_m)-s(f_n,\sigma_m)|<\mu
 \end{equation}
 Ma allora per ogni $\mu>0$ esistono $N$ e $M$ tali che per ogni $n\ge N$ e $m\ge M$ vale:
 \begin{gather*}
  |S(f,\sigma_m)-S(f_n,\sigma_m)|+ |s(f_n,\sigma_m)-s(f,\sigma_m)|+ |S(f_n,\sigma_m)-s(f_n,\sigma_m)|<\mu(2b-2a+1) \\
  \Longrightarrow |S(f,\sigma_m)-s(f,\sigma_m)|< \mu(2b-2a+1)
 \end{gather*}
 Quindi scegliendo $\mu=\varepsilon/(2b-2a+1)$ ottengo che per ogni $m\ge M$ vale:
 \begin{equation*}
  |S(f,\sigma_m)-s(f,\sigma_m)|< \varepsilon
 \end{equation*}
 Quindi $f$ è Riemann-integrabile su $[a,b]$, e in particolare dalla \cref{ViciniInIntervallo} si ottiene facilmente
 anche che
 \begin{equation}\label{IntegraleInIntervallo}
  \int_a^b{f(x)dx}=\lim_{n->\infty}\int_a^b{f_n(x)dx}
 \end{equation}
 Da quest'ultima relazione ottengo anche che $|f|$ è Riemann-integrabile in $[a,b]$, poichè in generale se
 $h$ è una funzione Riemann-integrabile in $[a,b]$ lo è anche $|h|$.\\
 Ora, dato che $|f_n|\le g$ per ogni $n\in\mathbb{N}$, passando al limite ottengo che $|f(x)|\le g(x)$ per ogni $x\in(0,\infty)$.
 Di conseguenza, dato che per quanto già detto $|f|$ è Riemann-integrabile in $[a,b]$, $|f|$ è Riemann-integrabile
 anche in $(0,\infty)$ perchè è non negativa.\\
 Ma dato che esiste l'integrale improprio di $|f|$ su $(0,\infty)$, allora esiste anche l'integrale improprio di $f$ su 
 $(0,\infty)$, e analogamente a quanto detto prima su un intervallo, vale proprio:
 \begin{equation*}
  \int_0^\infty{f(x)dx}=\lim_{n->\infty}\int_0^\infty{f_n(x)dx}
 \end{equation*}
\end{proof}

\begin{theorem}[Formula di Stirling per la funzione Gamma]
\label{StirlingGamma}
 La formula di Stirling offre un'approssimazione per $\Gamma(x+1)$:
 \begin{equation*}
  \lim_{x\to\infty}\frac{\Gamma(x+1)}{(x/e)^x\sqrt{2\pi x}}=1
 \end{equation*}
\end{theorem}
\begin{proof}
 Sostituendo $t=x(1+s\sqrt{2/x})$ nella definizione della funzione Gamma \cref{FunzioneGamma} ottengo:
 \begin{equation*}
  \begin{split}
   \Gamma(x+1) & = \int_0^\infty{e^{-t}t^{x}dt}\\
               & = \int_{-\sqrt{\frac{x}{2}}}^\infty{ e^{-x(1+s\sqrt{2/x})} x^x\left(1+s\sqrt{\frac{2}{x}}\right)^x \sqrt{2x} ds}\\
               & = e^{-x}x^x\sqrt{2x}\int_{-\sqrt{\frac{x}{2}}}^\infty{ \left[e^{-s\sqrt{2/x}} \left(1+s\sqrt{\frac{2}{x}}\right)\right]^x ds}\\
               & = e^{-x}x^x\sqrt{2x}\int_{-\sqrt{\frac{x}{2}}}^\infty{ e^{-s\sqrt{2x}+x\ln(1+s\sqrt{\frac{2}{x}})} ds}\\
               & = e^{-x}x^x\sqrt{2x}\int_{-\sqrt{\frac{x}{2}}}^\infty{ e^{-s^2(\frac{\sqrt{2x}}{s}-\frac{x}{s^2}\ln(1+s\sqrt{\frac{2}{x}}))} ds}
  \end{split}
 \end{equation*}
 Ora definisco per comodità $h(x,s)=\frac{\sqrt{2x}}{s}-\frac{x}{s^2}\ln(1+s\sqrt{\frac{2}{x}})$ 
 e mi concentro su quest'ultimo integrale, in particolare esso è uguale a:
 \begin{equation*}
  \int_{-\infty}^\infty{ f_x(s) ds}
 \end{equation*}
 dove 
 \begin{equation*}
  f_x(s)=\begin{cases}
          e^{-s^2h(x,s)}, & \mbox{se } -\sqrt{x/2}<s<\infty \\
          0, & \mbox{se } s\le -\sqrt{x/2}
         \end{cases}
 \end{equation*}



\end{proof}



