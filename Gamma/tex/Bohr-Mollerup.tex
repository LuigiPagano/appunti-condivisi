\begin{theorem}[Teorema di Bohr-Mollerup]
Esiste un'unica funzione $\Gamma :\mathbb{R}^{+}\rightarrow\mathbb{R} $ che rispetta le 3 seguenti proprietà:
\begin{itemize}
 \item $\Gamma(1)=1$
 \item $\Gamma(x+1)=\Gamma(x)$
 \item $\Gamma$ è log-convessa, cioè la funzione $\ln(\Gamma(x))$ è convessa
\end{itemize}
In particolare tale funzione è definita da $\Gamma(x)=\int_0^{\infty}{e^{-t}t^{x-1}dx}$.
\end{theorem}

\begin{proof}
Dimostro innanzitutto che la funzione $\Gamma(x)=\int_0^{\infty}{e^{-t}t^{x-1}dx}$ rispetta le tre condizioni:
\begin{itemize}
 \item \begin{equation}
       \Gamma(1)=\int_0^{\infty}{e^{-t}dt}=\left[-e^{-t}\right]_0^{\infty}=1
       \end{equation}
 \item Integrando per parti ottengo:
       \begin{equation} 
       \Gamma(x+1)=\int_0^{\infty}{e^{-t}t^xdt}=\left[-e^{-t}t^x\right]_0^{\infty}+\int_0^{\infty}xe^{-t}t^{x-1}dt=x\Gamma(x)
       \end{equation}
 \item $\Gamma$ è log-convessa se e solo se per ogni $0<\lambda < 1$ e per ogni $x,y \in \mathbb{R}^+$ vale 
       $\ln(\Gamma(\lambda x+(1-\lambda)y))\le \lambda \ln(\Gamma(x))+(1-\lambda)\ln(\Gamma(y))$.
       Per la disuguaglianza di Holder vale però che:
       \begin{equation}
       \begin{split}
       \int_0^{\infty}{(e^{-t}t^{x-1})^\lambda (e^{-t}t^{y-1})^{1-\lambda}dt} & \le 
       \left(\int_0^{\infty}{e^{-t}t^{x-1}dt}\right)^\lambda \left(\int_0^{\infty}{e^{-t}t^{y-1}}\right)^{1-\lambda}\\
       \Longrightarrow \int_0^{\infty}{e^{-t}t^{\lambda x +(1-\lambda)y}dt}  & \le 
       \left(\int_0^{\infty}{e^{-t}t^{x-1}dt}\right)^\lambda \left(\int_0^{\infty}{e^{-t}t^{y-1}}\right)^{1-\lambda}\\
       \Longrightarrow \Gamma(\lambda x+(1-\lambda)y) & \le \Gamma(x)^\lambda\Gamma(y)^{1-\lambda}\\
       \Longrightarrow \ln(\Gamma(\lambda x+(1-\lambda)y)) & \le \lambda\ln(\Gamma(x))+(1-\lambda)\ln(\Gamma(y))\\
       \end{split}
       \end{equation}
\end{itemize}
Dimostro ora che questa è l'unica funzione che rispetta le ipotesi. Sia $f$ una funzione che rispetta le tre condizioni, allora 

\begin{equation}
\begin{split}
 & \ln(\Gamma(n))-\ln(\Gamma(n-1)) \le \frac{\ln(\Gamma(n+x))-\ln(\Gamma(n))}{x} \le \ln(\Gamma(n+1))-\ln(\Gamma(n)) \\
 & \iff x \ln(n-1) \le \ln \left( \frac{\Gamma(n+x)}{\Gamma(n)} \right) \le x\ln(n)\\
 & \iff (n-1)^x \le \left( \frac{\Gamma(n+x)}{\Gamma(n)} \right) \le n^x \\
 & \iff (n-1)^x \le \left( \frac{\Gamma(x)(n+x-1)!}{(n-1)!} \right) \le n^x \\
 & \iff \left(1-\frac{1}{n}\right)^x \le \left( \frac{\Gamma(x)(n+x-1)!}{n^x(n-1)!} \right) \le 1\\
 & \Longrightarrow \lim_{n\rightarrow \infty} \left(1-\frac{1}{n}\right)^x \le \lim_{n\rightarrow \infty}  \frac{\Gamma(x)(n+x-1)!}{n^x(n-1)!} \le 1 \\
 & \Longrightarrow \lim_{n\rightarrow \infty} \frac{\Gamma(x)(n+x-1)!}{n^x(n-1)!}  = 1 \\
 & \Longrightarrow \Gamma(x) = \lim_{n\rightarrow \infty} \frac{n^x(n-1)!}{(n+x-1)!}  = \lim_{n\rightarrow \infty} \frac{n^xn!}{(n+x)!}
\end{split}
\end{equation}



\end{proof}
