\section{Teorema di Bohr-Mollerup}

\begin{theorem}[Teorema di Bohr-Mollerup]
\label{BohrMollerup}
Esiste un'unica funzione $f :\mathbb{R}^{+}\rightarrow\mathbb{R} $ che rispetta le tre seguenti proprietà:
\begin{itemize}
 \item $f(1)=1$
 \item $f(x+1)=xf(x)$
 \item $f$ è log-convessa, cioè la funzione $\ln(f(x))$ è convessa
\end{itemize}
In particolare tale funzione è definita da :
\begin{equation}
 \label{GammaDefinition}
 \Gamma(x)=\int_0^{\infty}{e^{-t}t^{x-1}dx}
\end{equation}
\end{theorem}
\begin{proof}
Dimostro innanzitutto che se esiste una funzione che rispetta le tre proprietà, allora è unica.

Sia $f$ una funzione che rispetta tali proprietà, allora per la seconda proprietà vale banalmente che:
\begin{equation}
 \label{Ricorsione}
 f(x+n)=(x+n-1)!f(x)\ \forall x\in \mathbb{R}^+, n\in\mathbb{N}
\end{equation}
dove $(x+n-1)!=(x+n-1)(x+n-2)\cdots(x+1)x$.\\
In particolare, dato che $f(1)=1$, ottengo anche che:
\begin{equation}
 \label{Naturali}
 f(n)=(n-1)!\ \forall n\in \mathbb{N}
\end{equation}

Sia ora $M(u,v)=\frac{\ln(f(u))-\ln(f(v))}{u-v}$ il rapporto incrementale
di $\ln(f(x))$ fra $u$ e $v$. Poichè $f$ è log-convessa, deve valere che $M(u,v)$ è crescente sia in $u$ che in $v$.
Quindi in particolare per ogni $0<x<1$ vale:
\begin{equation*}
 \begin{split}
  M(n,n-1) & \le M(n,n+x) \le M(n,n+1) \\
  \iff \ln(f(n))-\ln(f(n-1)) & \le \frac{\ln(f(n+x))-\ln(f(n))}{x} \le \ln(f(n+1))-\ln(f(n)) 
  \end{split}
\end{equation*}
Da cui, utilizzando la \cref{Ricorsione} e la \cref{Naturali}, ottengo che:
\begin{gather*}
  x \ln(n-1) \le \ln \left( \frac{f(n+x)}{f(n)} \right) \le x\ln(n)\\
  \iff (n-1)^x \le \left( \frac{f(n+x)}{f(n)} \right) \le n^x \\
  \iff (n-1)^x \le \left( \frac{f(x)(n+x-1)!}{(n-1)!} \right) \le n^x \\
  \iff \left(1-\frac{1}{n}\right)^x \le \left( \frac{f(x)(n+x-1)!}{n^x(n-1)!} \right) \le 1
\end{gather*}
E passando al limite:
 \begin{gather*}
  \lim_{n\rightarrow \infty} \left(1-\frac{1}{n}\right)^x \le \lim_{n\rightarrow \infty}  \frac{f(x)(n+x-1)!}{n^x(n-1)!} \le 1 \\
  \Longrightarrow \lim_{n\rightarrow \infty} \frac{f(x)(n+x-1)!}{n^x(n-1)!}  = 1 \\
  \Longrightarrow f(x) = \lim_{n\rightarrow \infty} \frac{n^x(n-1)!}{(n+x-1)!}=\lim_{n\rightarrow \infty} \frac{n^xn!}{(n+x)!}
 \end{gather*}
In particolare da quest'ultima equazione e dalla \cref{Ricorsione} ottengo che per ogni $x\in\mathbb{R}^+$ vale:
\begin{equation}
\label{GaussFormula}
 f(x)=\lim_{n\rightarrow \infty} \frac{n^xn!}{(n+x)!}
\end{equation}
E di conseguenza se esiste una funzione $f$ che rispetta le tre proprietà, essa deve essere un'unica, perchè ogni
funzione di questo tipo deve rispettare la relazione \cref{GaussFormula}.

Dimostro ora che tale funzione esiste e in particolare è 
$\Gamma(x)=\int_0^{\infty}{e^{-t}t^{x-1}dx}$.
\begin{lemma}
 $\Gamma(1)=1$
\end{lemma}
\begin{proof}
       \begin{equation*}
       \Gamma(1)=\int_0^{\infty}{e^{-t}dt}=\left[-e^{-t}\right]_0^{\infty}=1
       \end{equation*}
\end{proof}
\begin{lemma}
 $\Gamma(x+1)=x\Gamma(x)$
\end{lemma}
\begin{proof}
       Integrando per parti ottengo:
       \begin{equation*} 
       \Gamma(x+1)=\int_0^{\infty}{e^{-t}t^xdt}=\left[-e^{-t}t^x\right]_0^{\infty}+\int_0^{\infty}xe^{-t}t^{x-1}dt=x\Gamma(x)
       \end{equation*}
\end{proof}
\begin{lemma}
 \label{GammaLogConvessa}
 $\Gamma(x)$ è log-convessa.
\end{lemma}
\begin{proof}
       $\Gamma$ è log-convessa se e solo se per ogni $0<\lambda < 1$ e per ogni $x,y \in \mathbb{R}^+$ vale 
       $\ln(\Gamma(\lambda x+(1-\lambda)y))\le \lambda \ln(\Gamma(x))+(1-\lambda)\ln(\Gamma(y))$.
       Per la disuguaglianza di Holder vale però che:
       \begin{gather*}
       \int_0^{\infty}{(e^{-t}t^{x-1})^\lambda (e^{-t}t^{y-1})^{1-\lambda}dt} \le 
       \left(\int_0^{\infty}{e^{-t}t^{x-1}dt}\right)^\lambda \left(\int_0^{\infty}{e^{-t}t^{y-1}}\right)^{1-\lambda}\\
       \Longrightarrow \int_0^{\infty}{e^{-t}t^{\lambda x +(1-\lambda)y}dt} \le 
       \left(\int_0^{\infty}{e^{-t}t^{x-1}dt}\right)^\lambda \left(\int_0^{\infty}{e^{-t}t^{y-1}}\right)^{1-\lambda}\\
       \Longrightarrow \Gamma(\lambda x+(1-\lambda)y) \le \Gamma(x)^\lambda\Gamma(y)^{1-\lambda}\\
       \Longrightarrow \ln(\Gamma(\lambda x+(1-\lambda)y)) \le \lambda\ln(\Gamma(x))+(1-\lambda)\ln(\Gamma(y))\\
       \end{gather*}
\end{proof}

Quindi $\Gamma(x)$ per i tre lemmi precendenti è l'unica funzione che rispetta tutte e tre le proprietà richieste.
\end{proof}
