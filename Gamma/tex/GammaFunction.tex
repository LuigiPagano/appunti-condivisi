\section{Funzione Gamma}
In questa sezione definiamo la funzione Gamma e, dopo alcuni risultati introduttivi, dimostriamo il teorema di Bohr-Mollerup, che
in qualche senso mostra che la funzione Gamma è l'unico modo \emph{sensato} di estendere il fattoriale ai reali positivi.

Infine dimostreremo alcune definizioni equivalenti della funzione Gamma.

\begin{definition}[Funzione Gamma]\label{FunzioneGamma} 
	La funzione Gamma è definita da $\mathbb{R^+}$ in $\mathbb{R^+}$ come:
	\begin{equation*}
		\Gamma(x)=\int_0^{\infty}{e^{-t}t^{x-1}dt}
	\end{equation*}
\end{definition}

\begin{lemma}\label{GammaConverge}
	L'integrale mostrato in \cref{FunzioneGamma} per definire $\Gamma(x)$ converge per ogni $x>0$.
\end{lemma}
\begin{proof}
	La funzione sotto il segno di integrale è ovviamente integrabile su tutto $\mathbb{R^+}$, ed è positiva.
	
	Non resta che far vedere che l'integrale improprio esiste finito.
	
	Per quanto riguarda la convergenza in 0, è facile osservare che per ogni $t>0$ vale la disuguaglianza $e^{-t}t^{x-1}\le t^{x-1}$ 
	e perciò l'integrale in $0$ converge visto che $x>0$ per ipotesi.
	
	Riguardo la convergenza a $+\infty$, anche qui basta notare che definitivamente $e^{-t}t^{x-1}\le e^{-\frac t2}$
	e questo implica facilmente la convergenza a $+\infty$.
	
	Unendo le 2 convergenze dimostrate si ottiene proprio la convergenza dell'integrale su tutta la semiretta dei reali positivi (e quindi la 
	definizione della gamma è coerente su tutto $\mathbb{R^+}$.
\end{proof}


\begin{remark}[Valore di $\Gamma(1)$]\label{ValoreGamma1}
	Vale in particolare che:
	\begin{equation*}
		\Gamma(1)=\int_0^{\infty}{e^{-t}dt}=\left[-e^{-t}\right]_0^{\infty}=1
	\end{equation*}
\end{remark}


\begin{lemma}\label{FunzionaleGamma}
	La funzione Gamma rispetta la seguente identità per ogni $x\in\mathbb{R^+}$:
	\begin{equation*}
		\Gamma(x+1)=x\Gamma(x)
	\end{equation*}
\end{lemma}
\begin{proof}
	Integrando per parti ottengo:
	\begin{equation*} 
		\Gamma(x+1)=\int_0^{\infty}{e^{-t}t^xdt}=\left[-e^{-t}t^x\right]_0^{\infty}+\int_0^{\infty}xe^{-t}t^{x-1}dt=x\Gamma(x)
	\end{equation*}
\end{proof}

\begin{remark} \label{RicorsioneGamma}
	Per ogni $x\in\mathbb{R^+}$ e per ogni $n\in\mathbb{N}$, sfruttando il \cref{FunzionaleGamma}, vale la seguente relazione:
	\begin{equation}
		\Gamma(x+n)=(x+n-1)!\Gamma(x)
	\end{equation}
\end{remark}

\begin{remark} \label{ValoreGammaNaturali}
	Ponendo $x=0$ in \cref{RicorsioneGamma} ed utilizzando \cref{ValoreGamma1}, ottengo in particolare che per ogni $n\in\mathbb{N}$ vale:
	\begin{equation}
		\Gamma(n)=(n-1)!
	\end{equation}
\end{remark}


\begin{lemma}\label{GammaLogConvessa}
	La funzione Gamma è log-convessa.
\end{lemma}
\begin{proof}
	La funzione Gamma è log-convessa se e solo se per ogni $0<\lambda,\mu < 1$ tali che $\lambda+\mu=1$
	e per ogni $x,y \in \mathbb{R}^+$ vale:
	\begin{equation}\label{QuasiGammaLogConvessa}
	\begin{split} 
		\log \Gamma(\lambda x+\mu y )  & \le \lambda \log \Gamma(x) + \mu\log \Gamma( y )\\
		& \Updownarrow  \\
		\Gamma(\lambda x+\mu y ) & \le  \Gamma(x)^{\lambda}\Gamma( y )^{\mu}
	\end{split}\end{equation}
	Ora sostituendo la \cref{FunzioneGamma} nella \cref{QuasiGammaLogConvessa}, mi riconduco a dimostrare:
	\begin{equation*}
		\int_0^{\infty}{e^{-t}t^{\lambda x+\mu y-1}dx}=\int_0^{\infty}{(e^{-t}t^{x-1})^\lambda (e^{-t}t^{y-1})^{\mu}dt} \le 
		\left(\int_0^{\infty}{e^{-t}t^{x-1}dt}\right)^\lambda \left(\int_0^{\infty}{e^{-t}t^{y-1}}\right)^{\mu}
	\end{equation*}
	che è vera per la disuguaglianza di Holder.
\end{proof}
 
\begin{theorem}[Teorema di Bohr-Mollerup] \label{BohrMollerup}
	Esiste un'unica funzione $f :\mathbb{R}^{+}\to\mathbb{R} $ che rispetta le tre seguenti proprietà:
	\begin{itemize}
		\item $f(1)=1$
		\item $f(x+1)=xf(x)$
		\item $f$ è log-convessa, cioè la funzione $\ln(f(x))$ è convessa
	\end{itemize}
	In particolare tale funzione è la funzione Gamma già definita in \cref{FunzioneGamma}.
\end{theorem}
\begin{proof}
	La \cref{ValoreGamma1}, il \cref{FunzionaleGamma} e il \cref{GammaLogConvessa}, ci dicono già che la funzione
	Gamma rispetta tutte e tre le proprietà elencate. Vogliamo dimostrare che non ne esistono altre.\\
	Sia quindi $f:\mathbb{R^+}\to\mathbb{R}$ una funzione che rispetta le tre proprietà, allora analogamente a 
	\cref{RicorsioneGamma} ho che per ogni $x\in \mathbb{R^+}$ e per ogni $n\in\mathbb{N}$ vale:
	\begin{equation}\label{RicorsioneQuasiGamma}
		f(x+n)=(x+n-1)!f(x)
	\end{equation}
	e di conseguenza, dato che $f(1)=1$, per ogni $n\in\mathbb{N}$:
	\begin{equation}\label{ValoreQuasiGammaNaturali}
		f(n)=(n-1)!
	\end{equation}
	Sia ora $M(u,v)=\frac{\log(f(u))-\log(f(v))}{u-v}$ il rapporto incrementale
	di $\log(f(x))$ fra $u$ e $v$. Poichè $f$ è log-convessa, deve valere che $M(u,v)$ è crescente sia in $u$ che in $v$.
	Quindi in particolare per ogni $0<x<1$ vale:
	\begin{gather*}
		M(n,n-1) \le M(n,n+x) \le M(n,n+1) \\
		\Updownarrow \\
		\log(f(n))-\log(f(n-1)) \le \frac{\log(f(n+x))-\log(f(n))}{x} \le \log(f(n+1))-\log(f(n)) 
	\end{gather*}
	Da cui, utilizzando la \cref{RicorsioneQuasiGamma} e la \cref{ValoreQuasiGammaNaturali}, ottengo che:
	\begin{gather*}
		x \log(n-1) \le \log \left( \frac{f(n+x)}{f(n)} \right) \le x\log(n)\\
		\iff (n-1)^x \le \frac{f(n+x)}{f(n)} \le n^x \\
		\iff (n-1)^x \le  \frac{f(x)(n+x-1)!}{(n-1)!}  \le n^x \\
		\iff \left(1-\frac{1}{n}\right)^x \le  \frac{f(x)(n+x-1)!}{n^x(n-1)!} \le 1
	\end{gather*}
	E passando al limite:
	\begin{gather*}
		\lim_{n\to \infty} \left(1-\frac{1}{n}\right)^x \le \lim_{n\to \infty}  \frac{f(x)(n+x-1)!}{n^x(n-1)!} \le 1 \\
		\Longrightarrow \lim_{n\to \infty} \frac{f(x)(n+x-1)!}{n^x(n-1)!}  = 1 \\
		\Longrightarrow f(x) = \lim_{n\to \infty} \frac{n^x(n-1)!}{(n+x-1)!}=\lim_{n\to \infty} \frac{n^xn!}{(n+x)!}
	\end{gather*}
	In particolare da quest'ultima equazione e dalla \cref{RicorsioneQuasiGamma} ottengo che per ogni $x\in\mathbb{R}^+$ vale:
	\begin{equation}\label{QuasiGaussFormula}
		f(x)=\lim_{n\to \infty} \frac{n^xn!}{(n+x)!}
	\end{equation}
	E di conseguenza se esiste una funzione $f$ che rispetta le tre proprietà, essa deve essere un'unica, perchè ogni
	funzione di questo tipo deve rispettare la relazione \cref{QuasiGaussFormula}.
\end{proof}
 
\begin{corollary}[Formula di Gauss]\label{GaussFormula}
	Per ogni $x\in\mathbb{R^+}$ vale la seguente formula per la $\Gamma$:
\begin{equation}\label{GaussRealFormula}
	\Gamma(x)=\lim_{n\to \infty} \frac{n^xn!}{x(x+1)\cdots(x+n)}
\end{equation}
\end{corollary}
\begin{proof}
	Abbiamo già dimostrato che se una funzione rispetta le ipotesi del \cref{BohrMollerup}, allora vale la 
	\cref{QuasiGaussFormula} con $0<x\le 1$, ma la funzione Gamma rispetta tali ipotesi e quindi vale \cref{GaussRealFormula}
	per $0<x\le 1$.
	Resta da espandere tale risultato ad $x>1$.
	
	Dimostro che se vale la \cref{GaussRealFormula} per $x\in\mathbb{R^+}$ allora vale anche per $x+1$, visto che ho già che vale
	su tutto l'intervallo $(0,1]$ questo implica (per facile induzione) che \cref{GaussRealFormula} vale per ogni $x$ reale positivo.
	
	Sotto l'ipotesi che la formula valga per $x$ e sfruttando \cref{FunzionaleGamma} risulta vera la seguente catena di uguaglianze:
	\begin{equation*}\begin{split}
		\Gamma(x+1) & =x\Gamma(x)=x\lim_{n\to \infty} \frac{n^xn!}{x(x+1)\cdots(x+n)} = 
		\lim_{n\to \infty} \frac{n^{x+1}n!}{(x+1)\cdots(x+n)(x+n+1)}\cdot\frac{x+n+1}{n}\\
		& = \lim_{n\to \infty} \frac{n^{x+1}n!}{(x+1)\cdots(x+n)(x+n+1)}
	\end{split}\end{equation*}
	e questo è proprio quello che serviva per concludere la dimostrazione.
\end{proof}

\begin{corollary}[Formula di Weierstrass]\label{WeierstrassFormula}
	Per ogni $x\in\mathbb{R^+}$ vale la seguente formula per la $\Gamma$:
\begin{equation*}
	\Gamma(x)=\frac{e^{-\gamma x}}x\prod_{i=1}^{\infty} \frac{e^{\frac xi}}{1+\frac xi}
\end{equation*}
\end{corollary}
\begin{proof}
	Vale per ogni $x\in\mathbb{R^+},\ n\in\mathbb{N}$ la seguente identità (ottenuta solo attraverso manipolazioni algebriche):
	\begin{equation}\label{WeierIdentity}
		\frac{n^xn!}{x(x+1)\cdots (x+n)}=e^{x\left(\log{n}-\frac11-\frac12-\dots-\frac1n\right)}\frac1x\prod_{i=1}^n\frac{e^{\frac xi}}{1+\frac xi}
	\end{equation}
	
	Ricordiamo inoltre che la costante di Eulero-Mascheroni $\gamma\approx0.577$ è definita come:
	\begin{equation}\label{EuleroMascheroni}
		\gamma=\lim_{n\to\infty} \sum_{i=1}^n \frac1i -\log{n}
	\end{equation}
	
	Applicando l'operatore $\lim_{n\to\infty}$ a entrambi i membri della \cref{WeierIdentity} e sfruttando \cref{EuleroMascheroni,GaussFormula} ottengo:
	\begin{equation*}
		\Gamma(x)=\lim_{n\to \infty} \frac{n^xn!}{x(x+1)\cdots(x+n)}
		=\lim_{n\to\infty} e^{x\left(\log{n}-\frac11-\frac12-\dots-\frac1n\right)}\frac1x\prod_{i=1}^n\frac{e^{\frac xn}}{1+\frac xn}
		=\frac{e^{-\gamma x}}x\prod_{i=1}^{\infty} \frac{e^{\frac xi}}{1+\frac xi}
	\end{equation*}
	Che è proprio la formula di Weierstrass.
\end{proof}






