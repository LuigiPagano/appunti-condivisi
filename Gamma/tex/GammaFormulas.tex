\section{Alcune formule sulla Gamma}
Proponiamo ora alcune formule rispettate dalla funzione Gamma. Queste risultano utili sia per dimostrari ulteriori proprietà della
funzione qui studiata, sia per il calcolo di molti integrali definiti.

\begin{theorem}[Formula di riflessione] \label{Riflessione}
	Per $0<x<1$ reale vale:
	\begin{equation*}
		\Gamma(x)\Gamma(1-x)=\frac{\pi}{\sin(\pi x)}
	\end{equation*}
\end{theorem}
\begin{proof}
	Uso l'identità \cref{GaussFormula} per ottenere:
	\begin{equation}\begin{split} \label{MezzaRiflessione}
		\Gamma(x)\Gamma(1-x) & = \lim_{n\to\infty} \dfrac{n^xn!}{x(x+1)\cdots (x+n)} \cdot 
		\dfrac{n^{1-x}n!}{(1-x)(1-x+1)\cdots (1-x+n)}\\
		& =\lim_{n\to\infty} \frac{n}{n+1-x} \cdot \frac{1}{x} \cdot \prod_{k=1}^{n}\dfrac{k^2}{k^2-x^2} \\
		& =\left(\lim_{n\to\infty} \frac{n}{n+1-x} \right) \cdot 
		\left( \lim_{n\to\infty} x \prod_{k=1}^{n}\left(1-\frac{x^2}{k^2}\right) \right)^{-1}
	\end{split}\end{equation}

	Ed ora ricordando la formula come prodotto infinito del seno:
	\begin{equation*}
		\sin(\pi x)=\pi x \prod_{k=0}^{\infty} \left(1-\frac{x^2}{k^2}\right)
	\end{equation*}
	ottengo che l'ultimo membro della \cref{MezzaRiflessione} risulta $\frac{\pi}{\sin(\pi x)}$, che è la tesi del teorema.
\end{proof}

\begin{remark}[Valore di $\Gamma\left(\frac12\right)$]
	Ponendo $x=1$ nella formula di riflessione appena ottenuta si giunge a $\Gamma\left(\frac12\right)^2=\pi$. 
	Ricordando che la $\Gamma$ è sempre positiva sui reali positivi, ne deduco $\Gamma\left(\frac12\right)=\sqrt\pi$.
\end{remark}


\begin{theorem}\label{gf:Moltiplicazione}
	Fissato $m\in\mathbb{N}$ e $x\in\mathbb{R^+}$ vale:
	\begin{equation*}
		\Gamma(x)\Gamma\left(x+\frac 1m\right)\Gamma\left(x+\frac 2m\right)\cdots \Gamma\left(x+\frac {m-1}m\right)=
		(2\pi)^{\frac{m-1}2}\cdot m^{\frac12-mx}\cdot\Gamma(mx)
	\end{equation*}
\end{theorem}
\begin{proof}
	Sfruttando \cref{GaussFormula} e manipolando algebricamente l'espressione ottengo:
	\begin{equation}\begin{split}\label{gf:ContMolt}
		&\Gamma(x)\Gamma\left(x+\frac 1m\right)\Gamma\left(x+\frac 2m\right)\cdots \Gamma\left(x+\frac {m-1}m\right) =\\
		&=\lim_{n\to\infty} \prod_{i=0}^{m-1}\frac{n^{x+\frac im}n!}
		{\left(x+\frac im\right)\left(x+\frac im\right)\cdots \left(x+n+\frac im\right)}\\
		&=\lim_{n\to\infty}\frac{n^{mx}n^{\frac{m-1}2}(n!)^m}{\frac 1{m^{mn+1}}\cdot mx(mx+1)(mx+2)\cdots (mx+mn)}\cdot 
		\frac1 {\left(x+\frac 1m+n\right)\cdots \left(x+\frac {m-1}m+n\right)}\\
		&=\lim_{n\to\infty}\left(\frac{(mn)^{mx}mn!}{mx(mx+1)\cdots(mx+mn)}\right)m^{\frac12-mx}
		\left(\frac{n!^m}{(mn)!}\cdot\frac{m^{mn+\frac12}n^{\frac{m-1}2}}
		{\left(x+\frac 1m+n\right)\cdots \left(x+\frac {m-1}m+n\right)}\right)\\
		&=\Gamma(mx)m^{\frac12-mx} \cdot \lim_{n\to\infty} \left( \frac{n!^m}{(mn)!}\cdot 
		\frac{m^{mn+\frac12}}{n^{\frac{m-1}2}}\right)\cdot 
		\lim_{n\to\infty}\left( \left( 1+\frac{x+\frac{1}m}{n}\right)^{-1}
		\cdots \left( 1+\frac{x+\frac{m-1}m}{n}\right)^{-1}\right) \\
		&=\Gamma(mx)m^{\frac12-mx} \cdot \lim_{n\to\infty}\frac{n!^m}{(mn)!}\cdot \frac{m^{mn+\frac12}}{n^{\frac{m-1}2}}
	\end{split}\end{equation}
	
	Inoltre, applicando \cref{f:StirlingNaturali}, ho che (assumendo $m$ costante):
	\begin{equation}\label{gf:StirlingApplic}
		\frac{n!^m}{(mn)!}\sim\frac{\left(\sqrt{2\pi n}\right)^m\left(\frac ne\right)^{mn}}{\sqrt{2\pi mn}\left(\frac{mn}e\right)^{mn}}
		=\frac{(2\pi n)^{\frac{m-1}2}}{m^{mn+\frac12}}\Rightarrow 
		\lim_{n\to\infty}\frac{n!^m}{(mn)!}\cdot \frac{m^{mn+\frac12}}{n^{\frac{m-1}2}}=(2\pi)^{\frac{m-1}2}
	\end{equation}
	
	Sostituendo \cref{gf:StirlingApplic} in \cref{gf:ContMolt} ottengo la tesi del teorema.
\end{proof}

\begin{corollary}[Formula di duplicazione] \label{Duplicazione}
	Ponendo $m=2$ in \cref{gf:Moltiplicazione} ottengo che per $x>0$ reale vale:
	\begin{equation*}
		\Gamma(x)\Gamma\left(x+\frac12\right)=2^{1-2x}\sqrt{\pi}\Gamma(2x)
	\end{equation*}
\end{corollary}

\begin{remark}[Valore di $\Gamma\left(\frac12\right)$]
	Ponendo $x=\frac12$ nella formula di duplicazione appena ottenuta si giunge a 
	\begin{equation*}
		\Gamma\left(\frac12\right)\Gamma(1)=\sqrt{\pi}\Gamma(1) \to \Gamma\left(\frac12\right)=\sqrt{\pi}
	\end{equation*}
\end{remark}

\begin{remark}[Integrale di Gauss] \label{GaussIntegral}
	Sostituendo $t=s^2$ nella \cref{FunzioneGamma}, ottengo:
	\begin{equation}
		\Gamma(x)=2\int_0^{\infty}{s^{2x-1}e^{-s^2}ds}
	\end{equation}
	Da cui, ponendo $x=\frac{1}{2}$:
	\begin{equation*}
		\sqrt{\pi}=\Gamma\left(\frac{1}{2}\right)=2\int_0^{\infty}{e^{-s^2}ds}\to \int_{-\infty}^{\infty}{e^{-s^2}ds}=\sqrt{\pi}
	\end{equation*}
\end{remark}



