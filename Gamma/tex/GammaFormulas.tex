\begin{theorem}[Formula di riflessione]
\label{Riflessione}
Per $0<x<1$ reale vale:
\begin{equation*}
	\Gamma(x)\Gamma(1-x)=\frac{\pi}{\sin(\pi x)}
\end{equation*}

\end{theorem}

\begin{proof}
Uso l'identità \cref{GaussFormula} per ottenere:
\begin{equation}
\label{MezzaRiflessione}
\begin{split}
\Gamma(x)\Gamma(1-x) & = \lim_{n\to\infty} \dfrac{n^xn!}{x(x+1)\cdots (x+n)} \cdot \dfrac{n^{1-x}n!}{(1-x)(1-x+1)\cdots (1-x+n)}\\
 & =\lim_{n\to\infty} \frac{n}{n+1-x} \cdot \frac{1}{x} \cdot \prod_{k=1}^{n}\dfrac{k^2}{k^2-x^2} \\
  & =\left(\lim_{n\to\infty} \frac{n}{n+1-x} \right) \cdot \left( \lim_{n\to\infty} x \prod_{k=1}^{n}\left(1-\frac{x^2}{k^2}\right) \right)^{-1}
\end{split}
\end{equation}

Ed ora ricordando la formula come prodotto infinito del seno:
\begin{equation*}
	\sin(\pi x)=\pi x \prod_{k=0} \left(1-\frac{x^2}{k^2}\right)
\end{equation*}

Ottengo che l'ultimo membro della \cref{MezzaRiflessione} risulta $\frac{\pi}{\sin(\pi x)}$, che è la tesi del teorema.


\end{proof}

\begin{remark}[Valore di $\Gamma\left(\frac12\right)$]
Ponendo $x=1$ nella formula di riflessione appena ottenuta si giunge a $\Gamma\left(\frac12\right)^2=\pi$. 
Ricordando che la $\Gamma$ è sempre positiva sui reali positivi, ne deduco $\Gamma\left(\frac12\right)=\sqrt\pi$.
\end{remark}


\begin{theorem}[Formula di duplicazione]
\label{Duplicazione}
Per $0<x$ reale vale:
\begin{equation*}
	\Gamma(x)\Gamma\left(x+\frac12\right)=2^{1-2x}\sqrt{\pi}\Gamma(2x)
\end{equation*}

\end{theorem}

\begin{proof}
Uso l'identità \cref{GaussFormula} per ottenere:
\begin{equation}
\label{MezzaDuplicazione}
\begin{split}
\Gamma(x)\Gamma\left(x+\frac12\right) = \lim_{n\to\infty} 
\dfrac{n^xn!}{x(x+1)\cdots (x+n)} \cdot \dfrac{n^{x+\frac12} n!}{\left(x+\frac12\right)\left(x+\frac32\right)\cdots \left(x+\frac{2n+1}2\right)}\\
 =\lim_{n\to\infty} \dfrac{(2n)^{2x}}{2^{2x}}\sqrt{n}\cdot (2n)!\cdot\binom{2n}{n}^{-1}\cdot \dfrac{2^{2n+1}}{(2x)(2x+1)\cdots(2x+2n)}\cdot \dfrac{1}{x+\frac12+n} \\
 =2^{1-2x} \left(\lim_{n\to\infty} \dfrac{ (2n)^{2x}(2n)! }{ (2x)(2x+1)\cdots(2x+2n) }\right) 
  \left( \lim_{n\to\infty} \frac{2^{2n}\sqrt{n}}{x+\frac12+n} \binom{2n}{n}^{-1}\right)
\end{split}
\end{equation}

Ricordando ancora la \cref{GaussFormula} si nota che il primo dei due limiti vale $\Gamma(2x)$.\\
Mentre per il secondo basta sfruttare la stima asintotica:
\begin{equation*}
	\binom{2n}{n}\sim \frac{4^n}{\sqrt{\pi n }}
\end{equation*}
per avere che è uguale a $\sqrt{\pi}$.\\
Sostituendo i risultati ottenuti nella \cref{MezzaDuplicazione} si ottiene la formula di duplicazione.
\end{proof}

\begin{remark}[Valore di $\Gamma\left(\frac12\right)$]
Ponendo $x=\frac12$ nella formula di duplicazione appena ottenuta si giunge a 
\begin{equation*}
  \Gamma\left(\frac12\right)\Gamma(1)=\sqrt{\pi}\Gamma(1) \to \Gamma\left(\frac12\right)=\sqrt{\pi}
\end{equation*}
\end{remark}

\begin{remark}[Integrale di Gauss]
 \label{GaussIntegral}
 Sostituendo $t=s^2$ nella \cref{GammaDefinition}, ottengo:
 \begin{equation}
  \Gamma(x)=2\int_0^{\infty}{s^{2x-1}e^{-s^2}ds}
 \end{equation}
 Da cui, ponendo $x=\frac{1}{2}$:
 \begin{equation*}
  \sqrt{\pi}=\Gamma\left(\frac{1}{2}\right)=2\int_0^{\infty}{e^{-s^2}ds}\to \int_{-\infty}^{\infty}{e^{-s^2}ds}=\sqrt{\pi}
 \end{equation*}
\end{remark}



