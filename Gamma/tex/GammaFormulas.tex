\begin{theorem}[Formula di riflessione]
\label{Riflessione}
Dimostro che $\forall 0<x<1$ (in realtà vale per ogni $x$ complesso tranne gli interi non positivi) risulta:
\begin{equation}
	\Gamma(x)\Gamma(1-x)=\frac{\pi}{\sin(\pi x)}
\end{equation}

\end{theorem}

\begin{proof}
Uso l'identità \ref{GaussFormula} per ottenere:
\begin{equation}
\label{MezzaRiflessione}
\begin{split}
\Gamma(x)\Gamma(1-x) & = \lim_{n\to\infty} \dfrac{n^xn!}{x(x+1)\cdots (x+n)} \cdot \dfrac{n^{1-x}n!}{(1-x)(1-x+1)\cdots (1-x+n)}\\
 & =\lim_{n\to\infty} \frac{n}{n+1-x} \cdot \frac{1}{x} \cdot \prod_{k=1}^{n}\dfrac{k^2}{k^2-x^2} = \\
  & =\left(\lim_{n\to\infty} \frac{n}{n+1-x} \right) \cdot \left( \lim_{n\to\infty} x \prod_{k=1}^{n}\left(1-\frac{x^2}{k^2}\right) \right)^{-1}
\end{split}
\end{equation}

Ed ora ricordando la formula come prodotto infinito del seno:
\begin{equation}
	\sin(\pi x)=\pi x \prod_{k=0} \left(1-\frac{x^2}{k^2}\right)
\end{equation}

Ottengo che l'ultimo membro della \eqref{MezzaRiflessione} risulta $\frac{\pi}{\sin(\pi x)}$, che è la tesi del teorema.


\end{proof}
