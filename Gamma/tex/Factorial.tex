\section{Approssimazione del fattoriale}
Questa sezione è interamente dedicata ad un risultato che riteniamo preliminare allo studio della funzione Gamma:
l'approssimazione di Stirling.

Questo risultato, oltre ad essere usato successivamente per ottenere vari risultati sulla Gamma, può avvicinare il lettore
all'idea che la Gamma nasce come estensione ``esatta'' del fattoriale a tutti i numeri reali positivi.

\begin{lemma}[Prodotto di Wallis]\label{f:WallisProduct}
	Vale la seguente identità:
	\begin{equation*}
		\frac{\pi}{2}=\prod_{n=1}^\infty\frac{(2n)^2}{(2n-1)(2n+1)}
	\end{equation*}
\end{lemma}
\begin{proof}
	Definisco
	\begin{equation*}
		I_n=\int_0^\pi \sin^n{x} dx
	\end{equation*}
	Integrando per parti ottengo che
	\begin{equation*}
		I_n=\frac{n-1}{n}\cdot I_{n-2}
	\end{equation*}
	In particolare ho che $I_0=\pi$ e $I_1=2$, da cui:
	\begin{gather*}
		I_{2n}=\frac{(2n-1)!!}{(2n)!!}\cdot\pi\\
		I_{2n+1}=\frac{(2n)!!}{(2n+1)!!}\cdot 2
	\end{gather*}
	Dato che $\sin^{n+1}x\le \sin^n x\le \sin^{n-1} x$ per ogni $x\in\mathbb{R}$, ho che $I_{n+1}\le I_n\le I_{n-1}$,
	da cui dividendo per $I_{n+1}$:
	\begin{gather*}
		1\le \frac{I_n}{I_{n+1}}\le \frac{I_{n-1}}{I_{n+1}}=1+\frac{1}{n}\\
		\Longrightarrow \lim_{n\to\infty}\frac{I_n}{I_{n+1}}=1\\
		\Longrightarrow \lim_{n\to\infty}\frac{(2n-1)!!}{(2n)!!}\cdot\frac{(2n+1)!!}{(2n)!!}\cdot\frac{\pi}{2}=1\\
		\Longrightarrow \frac{\pi}{2}=\lim_{n\to\infty}\frac{\left[(2n)!!\right]^2}{(2n-1)!!(2n+1)!!}=\prod_{n=1}^\infty\frac{(2n)^2}{(2n-1)(2n+1)}
	\end{gather*}
\end{proof}


\begin{theorem}[Approssimazione di Stirling per il fattoriale]\label{f:StirlingNaturali}
	Riportiamo una dimostrazione dell'approssimazione di Stirling sui numeri naturali (che è una caso
	particolare di \cref{StirlingGamma})
	\begin{equation*}
		\lim_{n\to\infty}{\frac{n!}{(n/e)^n\sqrt{2\pi n}}}=1
	\end{equation*}
\end{theorem}
\begin{proof}
	Dimostriamo innanzitutto che esiste finito il seguente limite:
	\begin{equation}\label{f:EsisteLimite}
		\lim_{n\to\infty}{\frac{n!}{(n/e)^n\sqrt{n}}}
	\end{equation}
	cioè, passando al logaritmo, che esiste il limite:
	\begin{equation}\label{f:EsisteLimiteLogaritmo}
		\lim_{n\to\infty}{\sum_{k=1}^{n}\ln k-n\ln n+n-\frac{1}{2}\ln n}
	\end{equation}
	
	Vale la seguente identità:
	\begin{equation*}
		\sum_{k=1}^{n}\ln k=n\ln n-\sum_{k=1}^{n-1}{k\left(\ln(k+1)-\ln k \right)}=n\ln n-\sum_{k=1}^{n-1}{k\ln\left(1+\frac{1}{k}\right)}
	\end{equation*}
	da cui, sostituendo il logaritmo con il suo sviluppo di Taylor, ottengo:
	\begin{equation*}
	\begin{split}
		\sum_{k=1}^{n}\ln k	& =n\ln n-\sum_{k=1}^{n-1}{k\left(\frac{1}{k}-\frac{1}{2k^2}+\bigO\left(\frac{1}{k^3}\right) \right)}\\
							& =n\ln n-\sum_{k=1}^{n-1}\left(1-\frac{1}{2k}+\bigO\left(\frac{1}{k^2}\right)  \right)\\
							& =n\ln n-n+1+\frac{1}{2}\sum_{k=1}^{n-1}\frac{1}{k}+\sum_{k=1}^{n-1}\bigO\left(\frac{1}{k^2}\right)
	\end{split}
	\end{equation*}
	Ora sfruttando che $\sum_{k=1}^{n-1}\frac{1}{n}-\ln n\to \gamma$ e che $\sum_{k=1}^{n-1}\bigO\left(\frac{1}{k^2}\right)\to L$
	dato che converge assolutamente, ottengo:
	\begin{equation}\label{f:QuasiStirlingNaturali}
		\sum_{k=1}^{n}\ln k=n\ln n-n+\frac{1}{2}\ln n + \left(\frac{\gamma}2 + L\right)
	\end{equation}
	Che passando al limite è proprio equivalente a \cref{f:EsisteLimiteLogaritmo}.
	
	Dimostriamo ora che il valore del limite di \cref{f:EsisteLimite} è proprio $\sqrt{2\pi}$.
	Grazie a \cref{f:WallisProduct} ho che:
	\begin{equation}\label{f:PiFactorial}
	\begin{split}
		\sqrt{\pi}	& =\sqrt{2}\prod_{n=1}^\infty\frac{2n}{\sqrt{(2n-1)(2n+1)}}\\
					& =\lim_{n\to\infty}\frac{(2n)!!}{(2n-1)!!\sqrt{n}}\\
					& =\lim_{n\to\infty}\frac{\left[(2n)!!\right]^2}{(2n)!\sqrt{n}}=\lim_{n\to\infty}\frac{2^{2n}\left(n!\right)^2}{(2n)!\sqrt{n}}\\
	\end{split}
	\end{equation}
	Definisco ora:
	\begin{equation*}
		a_n=\frac{n!}{(n/e)^n\sqrt{n}}
	\end{equation*}
	Poichè $\lim_{n\to\infty}{a_n}$ esiste vale facilmente che $\lim_{n\to\infty}a_n/a_{2n}=1$, quindi:
	\begin{equation*}
		\lim_{n\to\infty}{\frac{n!}{(n/e)^n\sqrt{n}}}=\lim_{n\to\infty}{a_n}=\lim_{n\to\infty}{\frac{a_n^2}{a_{2n}}}
	\end{equation*}
	Grazie alla \cref{f:PiFactorial}, ho però che:
	\begin{equation*}
	\begin{split}
		\lim_{n\to\infty}{\frac{a_n^2}{a_{2n}}}
		&=\lim_{n\to\infty}{\frac{\left(n!\right)^2}{(n/e)^{2n}n}\cdot \frac{(2n/e)^{2n}\sqrt{2n}}{(2n)!} }\\
		&=\lim_{n\to\infty}{\sqrt{2}\cdot\frac{2^{2n} \left(n!\right)^2}{(2n)!\sqrt{n}} }=\sqrt{2\pi}
	\end{split}
	\end{equation*}
	da cui la tesi.
\end{proof}
