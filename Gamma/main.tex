\documentclass[a4paper,12pt]{article}
\usepackage{stilebase}

\title{Appunti sulla funzione Gamma}
\author{Federico Glaudo \and Giada Franz}

\begin{document}

\maketitle
\clearpage

\tableofcontents

\begin{abstract}
Queste sono delle dispense che raccolgono i primi risultati che si ottengono
studiando la funzione Gamma e le funzioni a lei collegate.

Le dimostrazioni sono volutamente tutte elementari: basta la teoria che si
ha dopo un corso di analisi I per comprenderle tutte.

Questo documento lo abbiamo scritto sia per allenarci a scrivere in latex, sia
perchè è impossibile trovare una guida alla gamma che sia allo stesso tempo 
elementare, completa e in italiano.

Speriamo di esservi d'aiuto.
\end{abstract}

\section{Teorema di Bohr-Mollerup}

\begin{theorem}[Teorema di Bohr-Mollerup]
\label{BohrMollerup}
Esiste un'unica funzione $f :\mathbb{R}^{+}\rightarrow\mathbb{R} $ che rispetta le tre seguenti proprietà:
\begin{itemize}
 \item $f(1)=1$
 \item $f(x+1)=xf(x)$
 \item $f$ è log-convessa, cioè la funzione $\ln(f(x))$ è convessa
\end{itemize}
In particolare tale funzione è definita da :
\begin{equation}
 \label{GammaDefinition}
 \Gamma(x)=\int_0^{\infty}{e^{-t}t^{x-1}dx}
\end{equation}
\end{theorem}
\begin{proof}
Dimostro innanzitutto che se esiste una funzione che rispetta le tre proprietà, allora è unica.\\
Sia $f$ una funzione che rispetta tali proprietà, allora per la seconda proprietà vale banalmente che:
\begin{equation}
 \label{Ricorsione}
 f(x+n)=(x+n-1)!f(x)\ \forall x\in \mathbb{R}^+, n\in\mathbb{N}
\end{equation}
dove $(x+n-1)!=(x+n-1)(x+n-2)\cdots(x+1)x$.\\
In particolare, dato che $f(1)=1$, ottengo anche che:
\begin{equation}
 \label{Naturali}
 f(n)=(n-1)!\ \forall n\in \mathbb{N}
\end{equation}
Sia ora $M(u,v)=\frac{\ln(f(u))-\ln(f(v))}{u-v}$ il rapporto incrementale
di $\ln(f(x))$ fra $u$ e $v$. Poichè $f$ è log-convessa, deve valere che $M(u,v)$ è crescente sia in $u$ che in $v$.
Quindi in particolare per ogni $0<x<1$ vale:
\begin{equation*}
 \begin{split}
  M(n,n-1) & \le M(n,n+x) \le M(n,n+1) \\
  \iff \ln(f(n))-\ln(f(n-1)) & \le \frac{\ln(f(n+x))-\ln(f(n))}{x} \le \ln(f(n+1))-\ln(f(n)) 
  \end{split}
\end{equation*}
Da cui, utilizzando la \cref{Ricorsione} e la \cref{Naturali}, ottengo che:
\begin{gather*}
  x \ln(n-1) \le \ln \left( \frac{f(n+x)}{f(n)} \right) \le x\ln(n)\\
  \iff (n-1)^x \le \left( \frac{f(n+x)}{f(n)} \right) \le n^x \\
  \iff (n-1)^x \le \left( \frac{f(x)(n+x-1)!}{(n-1)!} \right) \le n^x \\
  \iff \left(1-\frac{1}{n}\right)^x \le \left( \frac{f(x)(n+x-1)!}{n^x(n-1)!} \right) \le 1
\end{gather*}
E passando al limite:
 \begin{gather*}
  \lim_{n\rightarrow \infty} \left(1-\frac{1}{n}\right)^x \le \lim_{n\rightarrow \infty}  \frac{f(x)(n+x-1)!}{n^x(n-1)!} \le 1 \\
  \Longrightarrow \lim_{n\rightarrow \infty} \frac{f(x)(n+x-1)!}{n^x(n-1)!}  = 1 \\
  \Longrightarrow f(x) = \lim_{n\rightarrow \infty} \frac{n^x(n-1)!}{(n+x-1)!}=\lim_{n\rightarrow \infty} \frac{n^xn!}{(n+x)!}
 \end{gather*}
In particolare da quest'ultima equazione e dalla \cref{Ricorsione} ottengo che per ogni $x\in\mathbb{R}^+$ vale:
\begin{equation}
\label{GaussFormula}
 f(x)=\lim_{n\rightarrow \infty} \frac{n^xn!}{(n+x)!}
\end{equation}
E di conseguenza se esiste una funzione $f$ che rispetta le tre proprietà, essa deve essere un'unica, perchè ogni
funzione di questo tipo deve rispettare la relazione \cref{GaussFormula}.\\
Dimostro ora che tale funzione esiste e in particolare è 
$\Gamma(x)=\int_0^{\infty}{e^{-t}t^{x-1}dx}$.
\begin{lemma}
 $\Gamma(1)=1$
\end{lemma}
\begin{proof}
       \begin{equation*}
       \Gamma(1)=\int_0^{\infty}{e^{-t}dt}=\left[-e^{-t}\right]_0^{\infty}=1
       \end{equation*}
\end{proof}
\begin{lemma}
 $\Gamma(x+1)=x\Gamma(x)$
\end{lemma}
\begin{proof}
       Integrando per parti ottengo:
       \begin{equation*} 
       \Gamma(x+1)=\int_0^{\infty}{e^{-t}t^xdt}=\left[-e^{-t}t^x\right]_0^{\infty}+\int_0^{\infty}xe^{-t}t^{x-1}dt=x\Gamma(x)
       \end{equation*}
\end{proof}
\begin{lemma}
 \label{GammaLogConvessa}
 $\Gamma(x)$ è log-convessa.
\end{lemma}
\begin{proof}
       $\Gamma$ è log-convessa se e solo se per ogni $0<\lambda < 1$ e per ogni $x,y \in \mathbb{R}^+$ vale 
       $\ln(\Gamma(\lambda x+(1-\lambda)y))\le \lambda \ln(\Gamma(x))+(1-\lambda)\ln(\Gamma(y))$.
       Per la disuguaglianza di Holder vale però che:
       \begin{gather*}
       \int_0^{\infty}{(e^{-t}t^{x-1})^\lambda (e^{-t}t^{y-1})^{1-\lambda}dt} \le 
       \left(\int_0^{\infty}{e^{-t}t^{x-1}dt}\right)^\lambda \left(\int_0^{\infty}{e^{-t}t^{y-1}}\right)^{1-\lambda}\\
       \Longrightarrow \int_0^{\infty}{e^{-t}t^{\lambda x +(1-\lambda)y}dt} \le 
       \left(\int_0^{\infty}{e^{-t}t^{x-1}dt}\right)^\lambda \left(\int_0^{\infty}{e^{-t}t^{y-1}}\right)^{1-\lambda}\\
       \Longrightarrow \Gamma(\lambda x+(1-\lambda)y) \le \Gamma(x)^\lambda\Gamma(y)^{1-\lambda}\\
       \Longrightarrow \ln(\Gamma(\lambda x+(1-\lambda)y)) \le \lambda\ln(\Gamma(x))+(1-\lambda)\ln(\Gamma(y))\\
       \end{gather*}
\end{proof}
Quindi $\Gamma(x)$ per i tre lemmi precendenti è l'unica funzione che rispetta tutte e tre le proprietà richieste.
\end{proof}

\section{Funzione Beta}

\begin{definition}[Funzione Beta]\label{FunzioneBeta} 
	La funzione Beta è definita da $\mathbb{R^+}^2$ in $\mathbb{R^+}$ come:
	\begin{equation*}
		\mathrm{B} (x,y)=\int_0^1 t^{x-1}(1-t)^{y-1} \mathrm{d}t
	\end{equation*}
\end{definition}

\begin{lemma}\label{BetaSimmetrica} 
	La funzione Beta è simmetrica, in formule:
	\begin{equation*}
		\mathrm{B}(x,y)=\mathrm{B}(y,x)
	\end{equation*}
\end{lemma}
\begin{proof}
	Basta applicare la sostituzione $t'=1-t$ nell'integrale della definizione \cref{FunzioneBeta} 
	per ottenere esattamente la simmetria della funzione Beta.
\end{proof}

\begin{lemma}\label{BetaTrigonometrica} 
	La funzione Beta rispetta la seguente identità per ogni $x,y\in \mathbb{R^+}$:
	\begin{equation*}
		\mathrm{B} (x,y)=2\int_0^{\frac\pi2} \sin(u)^{2x-1}\cos(u)^{2y-1}\mathrm{d}u
	\end{equation*}
\end{lemma}
\begin{proof}
	Trasformo l'integrale della \cref{FunzioneBeta} con la sostituzione $t=\sin^2u$.\\
	Gli estremi d'integrazione diventano $0,\frac{\pi}2$ e risulta che $\mathrm{d}t=2\sin(u)\cos(u)du$. Unendo questi risultati ottengo:
	\begin{equation*}\begin{split}
		\mathrm{B}(x,y) & =\int_0^1 t^{x-1}(1-t)^{y-1} \mathrm{d}t = \int_0^{\frac\pi2} \sin(u)^{2(x-1)}\cos(u)^{2(y-1)}2\sin(u)\cos(u)\mathrm{d}u\\
		& = 2\int_0^{\frac\pi2} \sin(u)^{2x-1}\cos(u)^{2y-1}\mathrm{d}u
	\end{split}\end{equation*}
\end{proof}

\begin{lemma}\label{FunzionaleBeta1}
	Per ogni $x,y\in\mathbb{R^+}$ la Beta rispetta la seguente equazione funzionale:
	\begin{equation*}
		\mathrm{B}(x+1,y)+\mathrm{B}(x,y+1)=\mathrm{B}(x,y)
	\end{equation*}
\end{lemma}
\begin{proof}
	Basta applicare la definizione \cref{FunzioneBeta} ottenendo:
	\begin{equation*}
		\mathrm{B}(x+1,y)+\mathrm{B}(x,y+1)=\int_0^1 t^x(1-t)^{y-1}+t^{x-1}(1-t)^y\mathrm{d}t=
		\int_0^1 t^{x-1}(1-t)^{y-1}(t+(1-t))\mathrm{d}t=\mathrm{B}(x,y)
	\end{equation*}

\end{proof}

\begin{lemma}\label{FunzionaleBeta2}
	Per ogni $x,y\in\mathbb{R^+}$ la Beta rispetta anche l'equazione funzionale:
	\begin{equation*}
		y\cdot\mathrm{B}(x+1,y)=x\cdot\mathrm{B}(x,y+1)
	\end{equation*}
\end{lemma}
\begin{proof}
	Integrando per parti vale la seguente identità:
	\begin{equation}\label{QuasiFunzionaleBeta2}
		\int_0^1 t^{x}(1-t)^{y-1}\mathrm{d}t=\left[\frac{-t^x(1-t)^y}y\right]_0^1-\int_0^1\frac{-xt^{x-1}(1-t)^y}{y}\mathrm{d}t=
		\frac xy \int_0^1 t^{x-1}(1-t)^y\mathrm{d}t 
	\end{equation}
	Sostituendo la \cref{FunzioneBeta} nella \cref{QuasiFunzionaleBeta2} si ottiene:
	\begin{equation*}
		\mathrm{B}(x+1,y)=\frac xy \cdot \mathrm{B}(x,y+1)
	\end{equation*}
	che è equivalente alla tesi.
\end{proof}


\begin{corollary}\label{FunzionaleBeta3}
	Per ogni $x,y\in\mathbb{R^+}$ la Beta rispetta:
	\begin{equation*}
		B(x+1,y)=\frac{x}{x+y}\cdot B(x,y)
	\end{equation*}
\end{corollary}
\begin{proof}
	\Cref{FunzionaleBeta1,FunzionaleBeta2} implicano:
	\begin{equation}
		\left\{
		\begin{aligned}
			&\mathrm{B}(x+1,y) &+& &\mathrm{B}(x,y+1)  & =\mathrm{B}(x,y)\\
			y\cdot &\mathrm{B}(x+1,y) &-& x&\mathrm{B}(x,y+1) & =0
		\end{aligned}
		\right.
	\end{equation}
	Che è un sistema lineare nelle variabili $\mathrm{B}(x+1,y), \mathrm{B}(x,y+1)$ se si considera $\mathrm{B}(x,y)$ costante.\\
	Risolvendo nella variabile $\mathrm{B}(x+1,y)$ si ottiene esattamente la tesi del corollario.
\end{proof}

\begin{lemma}\label{BetaLogConvessa} 
	La funzione Beta è log-convessa in entrambi gli argomenti.
\end{lemma}
\begin{proof}
	Basta dimostrarlo per un argomento (considerando l'altro fissato) e poi grazie alla simmetria mostrata in \cref{BetaSimmetrica}
	si ottiene la tesi anche per l'altro.
	
	Resta quindi da dimostrare che per ogni $a,b,y\in\mathbb{R^+}$ e $\lambda,\mu\in\mathbb{R^+}$ che rispettano $\lambda+\mu=1$ vale la seguente:
	\begin{equation}\begin{split}\label{QuasiBetaLogConvessa}
		\log \mathrm{B}(\lambda a+\mu b, y )  & \le \lambda \log \mathrm{B}(a, y ) + \mu\log \mathrm{B}( b, y )\\
		& \Updownarrow  \\
		\mathrm{B}(\lambda a+\mu b, y ) & \le  \mathrm{B}(a,y)^{\lambda}\mathrm{B}(b, y )^{\mu}
	\end{split}\end{equation}
	Ora sostituisco nella \cref{QuasiBetaLogConvessa} la definizione \cref{FunzioneBeta} ottenendo che devo dimostrare:
	\begin{equation*}
		\int_{0}^1 t^{\lambda a+\mu b}(1-t)^y\mathrm{d}y=\int_{0}^1 \left(t^a(1-t)^y\right)^{\lambda}\left(t^b(1-t)^y\right)^{\mu}\mathrm{d}t \le
		\left(\int_{0}^1 t^a(1-t)^y\right)^{\lambda}\left(\int_{0}^1 t^b(1-t)^y\right)^{\mu}
	\end{equation*}
	e questa è vera per la disuguaglianza di Holder in forma integrale.

\end{proof}

\begin{theorem}\label{GammaBeta}
	Vale la seguente relazione tra funzione Gamma e Beta (con $x,y\in\mathbb{R}^+$):
	\begin{equation*}
		\mathrm{B}(x,y)=\frac{\Gamma(x)\Gamma(y)}{\Gamma(x+y)}
	\end{equation*}
\end{theorem}
\begin{proof}
	Fissato $y$ reale positivo, sia $f_y:\mathbb{R^+}\to\mathbb{R^+}$ la funzione che rispetta:
	\begin{equation}\label{QuasiGamma}
		f_y(x)=\frac{\mathrm{B}(x,y)\Gamma(x+y)}{\Gamma(y)}
	\end{equation}
	
	Dimostro che $f_y$ rispetta le ipotesi di \cref{BohrMollerup}.
	\begin{itemize}
		\item $f_y(1)=1$
		
			Vale, sfruttando \cref{QuasiGamma,FunzionaleGamma} che:
			\begin{equation}\label{QuasiGammaDiUno}
				f_y(1)=\frac{\mathrm{B}(1,y)\Gamma(1+y)}{\Gamma(y)}=\mathrm{B}(1,y)y
			\end{equation}
			
			Però dalla definizione \cref{FunzioneBeta} e dalla simmetria \cref{BetaSimmetrica} ho anche:
			\begin{equation}\label{BetaDiUno}
				\mathrm{B}(1,y)=\mathrm{B}(y,1)=\int_0^1 t^{y-1} \mathrm{d}t = \left[\frac{t^y}y\right]_0^1= \frac1y
			\end{equation}
			
			Sostituendo \cref{BetaDiUno} nella \cref{QuasiGammaDiUno} ottengo $f_y(1)=1$.
		\item $f_y$ è log-convessa
		
			Applicando la log convessità di Gamma e Beta (\cref{BetaLogConvessa,GammaLogConvessa}) 
			nella \cref{QuasiGamma} ho che $f_y$ è prodotto di funzioni log-convesse, perciò è essa stessa log-convessa.
		\item $f_y(x+1)=xf_y(x)$
		
			Applicando l'equazione funzionale della Gamma nella \cref{QuasiGamma} si ha facilmente:
			\begin{equation}\label{QuasiFunzionaleBeta}
				f_y(x+1)=\frac{\mathrm{B}(x+1,y)(x+y)\Gamma(x+y)}{\Gamma(y)}=(x+y) f_y(x) \frac{B(x+1,y)}{\mathrm{B}(x,y)}
			\end{equation}
			Ed ora applico \cref{FunzionaleBeta3} nella \cref{QuasiFunzionaleBeta} ottenendo quando desiderato:
			\begin{equation*}
				f_y(x+1)=(x+y) f_y(x) \frac{x}{x+y}=xf_y(x)
			\end{equation*}
	\end{itemize}
	Poichè $f_y$ rispetta tutte le ipotesi di \cref{BohrMollerup}, lo applico ottenendo che $f_y=\Gamma$.
	
	Di conseguenza, ricordando la definizione \cref{QuasiGamma} ottengo:
	\begin{equation*}
		\frac{\mathrm{B}(x,y)\Gamma(x+y)}{\Gamma(y)}=f_y(x)=\Gamma(x) \Longrightarrow \mathrm{B}(x,y)=\frac{\Gamma(x)\Gamma(y)}{\Gamma(x+y)}
	\end{equation*}
\end{proof}

\begin{remark}[Valore di $\Gamma\left(\frac12\right)$]
	Ponendo $x=y=1$ in \cref{GammaBeta} e sfruttando \cref{BetaTrigonometrica} ottengo:
	\begin{equation*}
		\mathrm{B}\left(\frac12,\frac12\right)=\dfrac{\Gamma\left(\frac12\right)^2}{\Gamma(1)}\Rightarrow 
		\Gamma\left(\frac12\right)^2=2\int_0^{\frac{\pi}2}\sin^0u\cos^0u\mathrm{d}u=\pi\Rightarrow \Gamma\left(\frac12\right)=\sqrt{\pi}
	\end{equation*}
\end{remark}
\begin{theorem}[Formula di riflessione]
\label{Riflessione}
Per $0<x<1$ reale vale:
\begin{equation*}
	\Gamma(x)\Gamma(1-x)=\frac{\pi}{\sin(\pi x)}
\end{equation*}

\end{theorem}

\begin{proof}
Uso l'identità \ref{GaussFormula} per ottenere:
\begin{equation}
\label{MezzaRiflessione}
\begin{split}
\Gamma(x)\Gamma(1-x) & = \lim_{n\to\infty} \dfrac{n^xn!}{x(x+1)\cdots (x+n)} \cdot \dfrac{n^{1-x}n!}{(1-x)(1-x+1)\cdots (1-x+n)}\\
 & =\lim_{n\to\infty} \frac{n}{n+1-x} \cdot \frac{1}{x} \cdot \prod_{k=1}^{n}\dfrac{k^2}{k^2-x^2} \\
  & =\left(\lim_{n\to\infty} \frac{n}{n+1-x} \right) \cdot \left( \lim_{n\to\infty} x \prod_{k=1}^{n}\left(1-\frac{x^2}{k^2}\right) \right)^{-1}
\end{split}
\end{equation}

Ed ora ricordando la formula come prodotto infinito del seno:
\begin{equation*}
	\sin(\pi x)=\pi x \prod_{k=0} \left(1-\frac{x^2}{k^2}\right)
\end{equation*}

Ottengo che l'ultimo membro della \eqref{MezzaRiflessione} risulta $\frac{\pi}{\sin(\pi x)}$, che è la tesi del teorema.


\end{proof}

\begin{theorem}[Formula di duplicazione]
\label{Duplicazione}
Per $0<x$ reale vale:
\begin{equation*}
	\Gamma(x)\Gamma\left(x+\frac12\right)=2^{1-2x}\sqrt{\pi}\Gamma(2x)
\end{equation*}

\end{theorem}

\begin{proof}
Uso l'identità \ref{GaussFormula} per ottenere:
\begin{equation}
\label{MezzaDuplicazione}
\begin{split}
\Gamma(x)\Gamma\left(x+\frac12\right) = \lim_{n\to\infty} 
\dfrac{n^xn!}{x(x+1)\cdots (x+n)} \cdot \dfrac{n^{x+\frac12} n!}{\left(x+\frac12\right)\left(x+\frac32\right)\cdots \left(x+\frac{2n+1}2\right)}\\
 =\lim_{n\to\infty} \dfrac{(2n)^{2x}}{2^{2x}}\sqrt{n}\cdot (2n)!\cdot\binom{2n}{n}^{-1}\cdot \dfrac{2^{2n+1}}{(2x)(2x+1)\cdots(2x+2n)}\cdot \dfrac{1}{x+\frac12+n} \\
 =2^{1-2x} \left(\lim_{n\to\infty} \dfrac{ (2n)^{2x}(2n)! }{ (2x)(2x+1)\cdots(2x+2n) }\right) 
  \left( \lim_{n\to\infty} \frac{2^{2n}\sqrt{n}}{x+\frac12+n} \binom{2n}{n}^{-1}\right)
\end{split}
\end{equation}

Ricordando ancora la \ref{GaussFormula} si nota che il primo dei due limiti vale $\Gamma(2x)$.\\
Mentre per il secondo basta sfruttare la stima asintotica:
\begin{equation*}
	\binom{2n}{n}\sim \frac{4^n}{\sqrt{\pi n }}
\end{equation*}
per avere che è uguale a $\sqrt{\pi}$.\\
Riunendo i risultati ottenuti si ottiene la formula di duplicazione.

\end{proof}


\end{document}

\makeindex