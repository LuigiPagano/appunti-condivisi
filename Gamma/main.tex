\documentclass[a4paper,12pt]{article}
\usepackage{stilebase}

\title{Appunti sulla funzione Gamma}
\author{Federico Glaudo \and Giada Franz}

\begin{document}

\maketitle
\clearpage


\begin{abstract}
	Queste sono delle dispense che raccolgono i primi risultati che si ottengono
	studiando la funzione Gamma e le funzioni a lei collegate.

	Le dimostrazioni sono volutamente tutte elementari: basta la teoria che si
	ha dopo un corso di analisi I per comprenderle tutte.

	Questo documento lo abbiamo scritto sia per allenarci a scrivere in latex, sia
	perchè è impossibile trovare una guida alla gamma che sia allo stesso tempo 
	elementare, completa e in italiano.
	
	Ovunque in questo documento tratteremo la Gamma ponendo come dominio i reali positivi, ma intendiamo chiarire
	che questa funzione può essere studiata come funzione dal piano complesso in se stesso, solo che non abbiamo i mezzi per farlo.

	Speriamo di esservi d'aiuto.
\end{abstract}
\clearpage

\tableofcontents
\clearpage

\section{Approssimazione del fattoriale}
Questa sezione è interamente dedicata ad un risultato che riteniamo preliminare allo studio della funzione Gamma:
l'approssimazione di Stirling.

Questo risultato, oltre ad essere usato successivamente per ottenere vari risultati sulla Gamma, può avvicinare il lettore
all'idea che la Gamma nasce come estensione ``esatta'' del fattoriale a tutti i numeri reali positivi.

\begin{lemma}[Prodotto di Wallis]\label{f:WallisProduct}
	Vale la seguente identità:
	\begin{equation*}
		\frac{\pi}{2}=\prod_{n=1}^\infty\frac{(2n)^2}{(2n-1)(2n+1)}
	\end{equation*}
\end{lemma}
\begin{proof}
	Definisco
	\begin{equation*}
		I_n=\int_0^\pi \sin^n{x} dx
	\end{equation*}
	Integrando per parti ottengo che
	\begin{equation*}
		I_n=\frac{n-1}{n}\cdot I_{n-2}
	\end{equation*}
	In particolare ho che $I_0=\pi$ e $I_1=2$, da cui:
	\begin{gather*}
		I_{2n}=\frac{(2n-1)!!}{(2n)!!}\cdot\pi\\
		I_{2n+1}=\frac{(2n)!!}{(2n+1)!!}\cdot 2
	\end{gather*}
	Dato che $\sin^{n+1}x\le \sin^n x\le \sin^{n-1} x$ per ogni $x\in\mathbb{R}$, ho che $I_{n+1}\le I_n\le I_{n-1}$,
	da cui dividendo per $I_{n+1}$:
	\begin{gather*}
		1\le \frac{I_n}{I_{n+1}}\le \frac{I_{n-1}}{I_{n+1}}=1+\frac{1}{n}\\
		\Longrightarrow \lim_{n\to\infty}\frac{I_n}{I_{n+1}}=1\\
		\Longrightarrow \lim_{n\to\infty}\frac{(2n-1)!!}{(2n)!!}\cdot\frac{(2n+1)!!}{(2n)!!}\cdot\frac{\pi}{2}=1\\
		\Longrightarrow \frac{\pi}{2}=\lim_{n\to\infty}\frac{\left[(2n)!!\right]^2}{(2n-1)!!(2n+1)!!}=\prod_{n=1}^\infty\frac{(2n)^2}{(2n-1)(2n+1)}
	\end{gather*}
\end{proof}


\begin{theorem}[Approssimazione di Stirling per il fattoriale]\label{f:StirlingNaturali}
	Riportiamo una dimostrazione dell'approssimazione di Stirling sui numeri naturali (che è una caso
	particolare di \cref{StirlingGamma})
	\begin{equation*}
		\lim_{n\to\infty}{\frac{n!}{(n/e)^n\sqrt{2\pi n}}}=1
	\end{equation*}
\end{theorem}
\begin{proof}
	Dimostriamo innanzitutto che esiste finito il seguente limite:
	\begin{equation}\label{f:EsisteLimite}
		\lim_{n\to\infty}{\frac{n!}{(n/e)^n\sqrt{n}}}
	\end{equation}
	cioè, passando al logaritmo, che esiste il limite:
	\begin{equation}\label{f:EsisteLimiteLogaritmo}
		\lim_{n\to\infty}{\sum_{k=1}^{n}\ln k-n\ln n+n-\frac{1}{2}\ln n}
	\end{equation}
	
	Vale la seguente identità:
	\begin{equation*}
		\sum_{k=1}^{n}\ln k=n\ln n-\sum_{k=1}^{n-1}{k\left(\ln(k+1)-\ln k \right)}=n\ln n-\sum_{k=1}^{n-1}{k\ln\left(1+\frac{1}{k}\right)}
	\end{equation*}
	da cui, sostituendo il logaritmo con il suo sviluppo di Taylor, ottengo:
	\begin{equation*}
	\begin{split}
		\sum_{k=1}^{n}\ln k	& =n\ln n-\sum_{k=1}^{n-1}{k\left(\frac{1}{k}-\frac{1}{2k^2}+\bigO\left(\frac{1}{k^3}\right) \right)}\\
							& =n\ln n-\sum_{k=1}^{n-1}\left(1-\frac{1}{2k}+\bigO\left(\frac{1}{k^2}\right)  \right)\\
							& =n\ln n-n+1+\frac{1}{2}\sum_{k=1}^{n-1}\frac{1}{k}+\sum_{k=1}^{n-1}\bigO\left(\frac{1}{k^2}\right)
	\end{split}
	\end{equation*}
	Ora sfruttando che $\sum_{k=1}^{n-1}\frac{1}{n}-\ln n\to \gamma$ e che $\sum_{k=1}^{n-1}\bigO\left(\frac{1}{k^2}\right)\to L$
	dato che converge assolutamente, ottengo:
	\begin{equation}\label{f:QuasiStirlingNaturali}
		\sum_{k=1}^{n}\ln k=n\ln n-n+\frac{1}{2}\ln n + \left(\frac{\gamma}2 + L\right)
	\end{equation}
	Che passando al limite è proprio equivalente a \cref{f:EsisteLimiteLogaritmo}.
	
	Dimostriamo ora che il valore del limite di \cref{f:EsisteLimite} è proprio $\sqrt{2\pi}$.
	Grazie a \cref{f:WallisProduct} ho che:
	\begin{equation}\label{f:PiFactorial}
	\begin{split}
		\sqrt{\pi}	& =\sqrt{2}\prod_{n=1}^\infty\frac{2n}{\sqrt{(2n-1)(2n+1)}}\\
					& =\lim_{n\to\infty}\frac{(2n)!!}{(2n-1)!!\sqrt{n}}\\
					& =\lim_{n\to\infty}\frac{\left[(2n)!!\right]^2}{(2n)!\sqrt{n}}=\lim_{n\to\infty}\frac{2^{2n}\left(n!\right)^2}{(2n)!\sqrt{n}}\\
	\end{split}
	\end{equation}
	Definisco ora:
	\begin{equation*}
		a_n=\frac{n!}{(n/e)^n\sqrt{n}}
	\end{equation*}
	Poichè $\lim_{n\to\infty}{a_n}$ esiste vale facilmente che $\lim_{n\to\infty}a_n/a_{2n}=1$, quindi:
	\begin{equation*}
		\lim_{n\to\infty}{\frac{n!}{(n/e)^n\sqrt{n}}}=\lim_{n\to\infty}{a_n}=\lim_{n\to\infty}{\frac{a_n^2}{a_{2n}}}
	\end{equation*}
	Grazie alla \cref{f:PiFactorial}, ho però che:
	\begin{equation*}
	\begin{split}
		\lim_{n\to\infty}{\frac{a_n^2}{a_{2n}}}
		&=\lim_{n\to\infty}{\frac{\left(n!\right)^2}{(n/e)^{2n}n}\cdot \frac{(2n/e)^{2n}\sqrt{2n}}{(2n)!} }\\
		&=\lim_{n\to\infty}{\sqrt{2}\cdot\frac{2^{2n} \left(n!\right)^2}{(2n)!\sqrt{n}} }=\sqrt{2\pi}
	\end{split}
	\end{equation*}
	da cui la tesi.
\end{proof}

\section{Funzione Gamma}
In questa sezione definiamo la funzione Gamma e, dopo alcuni risultati introduttivi, dimostriamo il teorema di Bohr-Mollerup, che
in qualche senso mostra che la funzione Gamma è l'unico modo \emph{sensato} di estendere il fattoriale ai reali positivi.

Infine dimostreremo alcune definizioni equivalenti della funzione Gamma.

\begin{definition}[Funzione Gamma]\label{FunzioneGamma} 
	La funzione Gamma è definita da $\mathbb{R^+}$ in $\mathbb{R^+}$ come:
	\begin{equation*}
		\Gamma(x)=\int_0^{\infty}{e^{-t}t^{x-1}dt}
	\end{equation*}
\end{definition}

\begin{lemma}\label{GammaConverge}
	L'integrale mostrato in \cref{FunzioneGamma} per definire $\Gamma(x)$ converge per ogni $x>0$.
\end{lemma}
\begin{proof}
	La funzione sotto il segno di integrale è ovviamente integrabile su tutto $\mathbb{R^+}$, ed è positiva.
	
	Non resta che far vedere che l'integrale improprio esiste finito.
	
	Per quanto riguarda la convergenza in 0, è facile osservare che per ogni $t>0$ vale la disuguaglianza $e^{-t}t^{x-1}\le t^{x-1}$ 
	e perciò l'integrale in $0$ converge visto che $x>0$ per ipotesi.
	
	Riguardo la convergenza a $+\infty$, anche qui basta notare che definitivamente $e^{-t}t^{x-1}\le e^{-\frac t2}$
	e questo implica facilmente la convergenza a $+\infty$.
	
	Unendo le 2 convergenze dimostrate si ottiene proprio la convergenza dell'integrale su tutta la semiretta dei reali positivi (e quindi la 
	definizione della gamma è coerente su tutto $\mathbb{R^+}$.
\end{proof}


\begin{remark}[Valore di $\Gamma(1)$]\label{ValoreGamma1}
	Vale in particolare che:
	\begin{equation*}
		\Gamma(1)=\int_0^{\infty}{e^{-t}dt}=\left[-e^{-t}\right]_0^{\infty}=1
	\end{equation*}
\end{remark}


\begin{lemma}\label{FunzionaleGamma}
	La funzione Gamma rispetta la seguente identità per ogni $x\in\mathbb{R^+}$:
	\begin{equation*}
		\Gamma(x+1)=x\Gamma(x)
	\end{equation*}
\end{lemma}
\begin{proof}
	Integrando per parti ottengo:
	\begin{equation*} 
		\Gamma(x+1)=\int_0^{\infty}{e^{-t}t^xdt}=\left[-e^{-t}t^x\right]_0^{\infty}+\int_0^{\infty}xe^{-t}t^{x-1}dt=x\Gamma(x)
	\end{equation*}
\end{proof}

\begin{remark} \label{RicorsioneGamma}
	Per ogni $x\in\mathbb{R^+}$ e per ogni $n\in\mathbb{N}$, sfruttando il \cref{FunzionaleGamma}, vale la seguente relazione:
	\begin{equation}
		\Gamma(x+n)=(x+n-1)!\Gamma(x)
	\end{equation}
\end{remark}

\begin{remark} \label{ValoreGammaNaturali}
	Ponendo $x=0$ in \cref{RicorsioneGamma} ed utilizzando \cref{ValoreGamma1}, ottengo in particolare che per ogni $n\in\mathbb{N}$ vale:
	\begin{equation}
		\Gamma(n)=(n-1)!
	\end{equation}
\end{remark}


\begin{lemma}\label{GammaLogConvessa}
	La funzione Gamma è log-convessa.
\end{lemma}
\begin{proof}
	La funzione Gamma è log-convessa se e solo se per ogni $0<\lambda,\mu < 1$ tali che $\lambda+\mu=1$
	e per ogni $x,y \in \mathbb{R}^+$ vale:
	\begin{equation}\label{QuasiGammaLogConvessa}
	\begin{split} 
		\log \Gamma(\lambda x+\mu y )  & \le \lambda \log \Gamma(x) + \mu\log \Gamma( y )\\
		& \Updownarrow  \\
		\Gamma(\lambda x+\mu y ) & \le  \Gamma(x)^{\lambda}\Gamma( y )^{\mu}
	\end{split}\end{equation}
	Ora sostituendo la \cref{FunzioneGamma} nella \cref{QuasiGammaLogConvessa}, mi riconduco a dimostrare:
	\begin{equation*}
		\int_0^{\infty}{e^{-t}t^{\lambda x+\mu y-1}dx}=\int_0^{\infty}{(e^{-t}t^{x-1})^\lambda (e^{-t}t^{y-1})^{\mu}dt} \le 
		\left(\int_0^{\infty}{e^{-t}t^{x-1}dt}\right)^\lambda \left(\int_0^{\infty}{e^{-t}t^{y-1}}\right)^{\mu}
	\end{equation*}
	che è vera per la disuguaglianza di Holder.
\end{proof}
 
\begin{theorem}[Teorema di Bohr-Mollerup] \label{BohrMollerup}
	Esiste un'unica funzione $f :\mathbb{R}^{+}\to\mathbb{R} $ che rispetta le tre seguenti proprietà:
	\begin{itemize}
		\item $f(1)=1$
		\item $f(x+1)=xf(x)$
		\item $f$ è log-convessa, cioè la funzione $\ln(f(x))$ è convessa
	\end{itemize}
	In particolare tale funzione è la funzione Gamma già definita in \cref{FunzioneGamma}.
\end{theorem}
\begin{proof}
	La \cref{ValoreGamma1}, il \cref{FunzionaleGamma} e il \cref{GammaLogConvessa}, ci dicono già che la funzione
	Gamma rispetta tutte e tre le proprietà elencate. Vogliamo dimostrare che non ne esistono altre.\\
	Sia quindi $f:\mathbb{R^+}\to\mathbb{R}$ una funzione che rispetta le tre proprietà, allora analogamente a 
	\cref{RicorsioneGamma} ho che per ogni $x\in \mathbb{R^+}$ e per ogni $n\in\mathbb{N}$ vale:
	\begin{equation}\label{RicorsioneQuasiGamma}
		f(x+n)=(x+n-1)!f(x)
	\end{equation}
	e di conseguenza, dato che $f(1)=1$, per ogni $n\in\mathbb{N}$:
	\begin{equation}\label{ValoreQuasiGammaNaturali}
		f(n)=(n-1)!
	\end{equation}
	Sia ora $M(u,v)=\frac{\log(f(u))-\log(f(v))}{u-v}$ il rapporto incrementale
	di $\log(f(x))$ fra $u$ e $v$. Poichè $f$ è log-convessa, deve valere che $M(u,v)$ è crescente sia in $u$ che in $v$.
	Quindi in particolare per ogni $0<x<1$ vale:
	\begin{gather*}
		M(n,n-1) \le M(n,n+x) \le M(n,n+1) \\
		\Updownarrow \\
		\log(f(n))-\log(f(n-1)) \le \frac{\log(f(n+x))-\log(f(n))}{x} \le \log(f(n+1))-\log(f(n)) 
	\end{gather*}
	Da cui, utilizzando la \cref{RicorsioneQuasiGamma} e la \cref{ValoreQuasiGammaNaturali}, ottengo che:
	\begin{gather*}
		x \log(n-1) \le \log \left( \frac{f(n+x)}{f(n)} \right) \le x\log(n)\\
		\iff (n-1)^x \le \frac{f(n+x)}{f(n)} \le n^x \\
		\iff (n-1)^x \le  \frac{f(x)(n+x-1)!}{(n-1)!}  \le n^x \\
		\iff \left(1-\frac{1}{n}\right)^x \le  \frac{f(x)(n+x-1)!}{n^x(n-1)!} \le 1
	\end{gather*}
	E passando al limite:
	\begin{gather*}
		\lim_{n\to \infty} \left(1-\frac{1}{n}\right)^x \le \lim_{n\to \infty}  \frac{f(x)(n+x-1)!}{n^x(n-1)!} \le 1 \\
		\Longrightarrow \lim_{n\to \infty} \frac{f(x)(n+x-1)!}{n^x(n-1)!}  = 1 \\
		\Longrightarrow f(x) = \lim_{n\to \infty} \frac{n^x(n-1)!}{(n+x-1)!}=\lim_{n\to \infty} \frac{n^xn!}{(n+x)!}
	\end{gather*}
	In particolare da quest'ultima equazione e dalla \cref{RicorsioneQuasiGamma} ottengo che per ogni $x\in\mathbb{R}^+$ vale:
	\begin{equation}\label{QuasiGaussFormula}
		f(x)=\lim_{n\to \infty} \frac{n^xn!}{(n+x)!}
	\end{equation}
	E di conseguenza se esiste una funzione $f$ che rispetta le tre proprietà, essa deve essere un'unica, perchè ogni
	funzione di questo tipo deve rispettare la relazione \cref{QuasiGaussFormula}.
\end{proof}
 
\begin{corollary}[Formula di Gauss]\label{GaussFormula}
	Per ogni $x\in\mathbb{R^+}$ vale la seguente formula per la $\Gamma$:
\begin{equation}\label{GaussRealFormula}
	\Gamma(x)=\lim_{n\to \infty} \frac{n^xn!}{x(x+1)\cdots(x+n)}
\end{equation}
\end{corollary}
\begin{proof}
	Abbiamo già dimostrato che se una funzione rispetta le ipotesi del \cref{BohrMollerup}, allora vale la 
	\cref{QuasiGaussFormula} con $0<x\le 1$, ma la funzione Gamma rispetta tali ipotesi e quindi vale \cref{GaussRealFormula}
	per $0<x\le 1$.
	Resta da espandere tale risultato ad $x>1$.
	
	Dimostro che se vale la \cref{GaussRealFormula} per $x\in\mathbb{R^+}$ allora vale anche per $x+1$, visto che ho già che vale
	su tutto l'intervallo $(0,1]$ questo implica (per facile induzione) che \cref{GaussRealFormula} vale per ogni $x$ reale positivo.
	
	Sotto l'ipotesi che la formula valga per $x$ e sfruttando \cref{FunzionaleGamma} risulta vera la seguente catena di uguaglianze:
	\begin{equation*}\begin{split}
		\Gamma(x+1) & =x\Gamma(x)=x\lim_{n\to \infty} \frac{n^xn!}{x(x+1)\cdots(x+n)} = 
		\lim_{n\to \infty} \frac{n^{x+1}n!}{(x+1)\cdots(x+n)(x+n+1)}\cdot\frac{x+n+1}{n}\\
		& = \lim_{n\to \infty} \frac{n^{x+1}n!}{(x+1)\cdots(x+n)(x+n+1)}
	\end{split}\end{equation*}
	e questo è proprio quello che serviva per concludere la dimostrazione.
\end{proof}

\begin{corollary}[Formula di Weierstrass]\label{WeierstrassFormula}
	Per ogni $x\in\mathbb{R^+}$ vale la seguente formula per la $\Gamma$:
\begin{equation*}
	\Gamma(x)=\frac{e^{-\gamma x}}x\prod_{i=1}^{\infty} \frac{e^{\frac xi}}{1+\frac xi}
\end{equation*}
\end{corollary}
\begin{proof}
	Vale per ogni $x\in\mathbb{R^+},\ n\in\mathbb{N}$ la seguente identità (ottenuta solo attraverso manipolazioni algebriche):
	\begin{equation}\label{WeierIdentity}
		\frac{n^xn!}{x(x+1)\cdots (x+n)}=e^{x\left(\log{n}-\frac11-\frac12-\dots-\frac1n\right)}\frac1x\prod_{i=1}^n\frac{e^{\frac xi}}{1+\frac xi}
	\end{equation}
	
	Ricordiamo inoltre che la costante di Eulero-Mascheroni $\gamma\approx0.577$ è definita come:
	\begin{equation}\label{EuleroMascheroni}
		\gamma=\lim_{n\to\infty} \sum_{i=1}^n \frac1i -\log{n}
	\end{equation}
	
	Applicando l'operatore $\lim_{n\to\infty}$ a entrambi i membri della \cref{WeierIdentity} e sfruttando \cref{EuleroMascheroni,GaussFormula} ottengo:
	\begin{equation*}
		\Gamma(x)=\lim_{n\to \infty} \frac{n^xn!}{x(x+1)\cdots(x+n)}
		=\lim_{n\to\infty} e^{x\left(\log{n}-\frac11-\frac12-\dots-\frac1n\right)}\frac1x\prod_{i=1}^n\frac{e^{\frac xn}}{1+\frac xn}
		=\frac{e^{-\gamma x}}x\prod_{i=1}^{\infty} \frac{e^{\frac xi}}{1+\frac xi}
	\end{equation*}
	Che è proprio la formula di Weierstrass.
\end{proof}







\section{Funzione Beta}

\begin{definition}[Funzione Beta]\label{FunzioneBeta} 
	La funzione Beta è definita da $\mathbb{R^+}^2$ in $\mathbb{R^+}$ come:
	\begin{equation*}
		\mathrm{B} (x,y)=\int_0^1 t^{x-1}(1-t)^{y-1} \mathrm{d}t
	\end{equation*}
\end{definition}

\begin{lemma}\label{BetaSimmetrica} 
	La funzione Beta è simmetrica, in formule:
	\begin{equation*}
		\mathrm{B}(x,y)=\mathrm{B}(y,x)
	\end{equation*}
\end{lemma}
\begin{proof}
	Basta applicare la sostituzione $t'=1-t$ nell'integrale della definizione \cref{FunzioneBeta} 
	per ottenere esattamente la simmetria della funzione Beta.
\end{proof}

\begin{lemma}\label{BetaTrigonometrica} 
	La funzione Beta rispetta la seguente identità per ogni $x,y\in \mathbb{R^+}$:
	\begin{equation*}
		\mathrm{B} (x,y)=2\int_0^{\frac\pi2} \sin(u)^{2x-1}\cos(u)^{2y-1}\mathrm{d}u
	\end{equation*}
\end{lemma}
\begin{proof}
	Trasformo l'integrale della \cref{FunzioneBeta} con la sostituzione $t=\sin^2u$.\\
	Gli estremi d'integrazione diventano $0,\frac{\pi}2$ e risulta che $\mathrm{d}t=2\sin(u)\cos(u)du$. Unendo questi risultati ottengo:
	\begin{equation*}\begin{split}
		\mathrm{B}(x,y) & =\int_0^1 t^{x-1}(1-t)^{y-1} \mathrm{d}t = \int_0^{\frac\pi2} \sin(u)^{2(x-1)}\cos(u)^{2(y-1)}2\sin(u)\cos(u)\mathrm{d}u\\
		& = 2\int_0^{\frac\pi2} \sin(u)^{2x-1}\cos(u)^{2y-1}\mathrm{d}u
	\end{split}\end{equation*}
\end{proof}

\begin{lemma}\label{FunzionaleBeta1}
	Per ogni $x,y\in\mathbb{R^+}$ la Beta rispetta la seguente equazione funzionale:
	\begin{equation*}
		\mathrm{B}(x+1,y)+\mathrm{B}(x,y+1)=\mathrm{B}(x,y)
	\end{equation*}
\end{lemma}
\begin{proof}
	Basta applicare la definizione \cref{FunzioneBeta} ottenendo:
	\begin{equation*}
		\mathrm{B}(x+1,y)+\mathrm{B}(x,y+1)=\int_0^1 t^x(1-t)^{y-1}+t^{x-1}(1-t)^y\mathrm{d}t=
		\int_0^1 t^{x-1}(1-t)^{y-1}(t+(1-t))\mathrm{d}t=\mathrm{B}(x,y)
	\end{equation*}

\end{proof}

\begin{lemma}\label{FunzionaleBeta2}
	Per ogni $x,y\in\mathbb{R^+}$ la Beta rispetta anche l'equazione funzionale:
	\begin{equation*}
		y\cdot\mathrm{B}(x+1,y)=x\cdot\mathrm{B}(x,y+1)
	\end{equation*}
\end{lemma}
\begin{proof}
	Integrando per parti vale la seguente identità:
	\begin{equation}\label{QuasiFunzionaleBeta2}
		\int_0^1 t^{x}(1-t)^{y-1}\mathrm{d}t=\left[\frac{-t^x(1-t)^y}y\right]_0^1-\int_0^1\frac{-xt^{x-1}(1-t)^y}{y}\mathrm{d}t=
		\frac xy \int_0^1 t^{x-1}(1-t)^y\mathrm{d}t 
	\end{equation}
	Sostituendo la \cref{FunzioneBeta} nella \cref{QuasiFunzionaleBeta2} si ottiene:
	\begin{equation*}
		\mathrm{B}(x+1,y)=\frac xy \cdot \mathrm{B}(x,y+1)
	\end{equation*}
	che è equivalente alla tesi.
\end{proof}


\begin{corollary}\label{FunzionaleBeta3}
	Per ogni $x,y\in\mathbb{R^+}$ la Beta rispetta:
	\begin{equation*}
		B(x+1,y)=\frac{x}{x+y}\cdot B(x,y)
	\end{equation*}
\end{corollary}
\begin{proof}
	\Cref{FunzionaleBeta1,FunzionaleBeta2} implicano:
	\begin{equation}
		\left\{
		\begin{aligned}
			&\mathrm{B}(x+1,y) &+& &\mathrm{B}(x,y+1)  & =\mathrm{B}(x,y)\\
			y\cdot &\mathrm{B}(x+1,y) &-& x&\mathrm{B}(x,y+1) & =0
		\end{aligned}
		\right.
	\end{equation}
	Che è un sistema lineare nelle variabili $\mathrm{B}(x+1,y), \mathrm{B}(x,y+1)$ se si considera $\mathrm{B}(x,y)$ costante.\\
	Risolvendo nella variabile $\mathrm{B}(x+1,y)$ si ottiene esattamente la tesi del corollario.
\end{proof}

\begin{lemma}\label{BetaLogConvessa} 
	La funzione Beta è log-convessa in entrambi gli argomenti.
\end{lemma}
\begin{proof}
	Basta dimostrarlo per un argomento (considerando l'altro fissato) e poi grazie alla simmetria mostrata in \cref{BetaSimmetrica}
	si ottiene la tesi anche per l'altro.
	
	Resta quindi da dimostrare che per ogni $a,b,y\in\mathbb{R^+}$ e $\lambda,\mu\in\mathbb{R^+}$ che rispettano $\lambda+\mu=1$ vale la seguente:
	\begin{equation}\begin{split}\label{QuasiBetaLogConvessa}
		\log \mathrm{B}(\lambda a+\mu b, y )  & \le \lambda \log \mathrm{B}(a, y ) + \mu\log \mathrm{B}( b, y )\\
		& \Updownarrow  \\
		\mathrm{B}(\lambda a+\mu b, y ) & \le  \mathrm{B}(a,y)^{\lambda}\mathrm{B}(b, y )^{\mu}
	\end{split}\end{equation}
	Ora sostituisco nella \cref{QuasiBetaLogConvessa} la definizione \cref{FunzioneBeta} ottenendo che devo dimostrare:
	\begin{equation*}
		\int_{0}^1 t^{\lambda a+\mu b}(1-t)^y\mathrm{d}y=\int_{0}^1 \left(t^a(1-t)^y\right)^{\lambda}\left(t^b(1-t)^y\right)^{\mu}\mathrm{d}t \le
		\left(\int_{0}^1 t^a(1-t)^y\right)^{\lambda}\left(\int_{0}^1 t^b(1-t)^y\right)^{\mu}
	\end{equation*}
	e questa è vera per la disuguaglianza di Holder in forma integrale.

\end{proof}

\begin{theorem}\label{GammaBeta}
	Vale la seguente relazione tra funzione Gamma e Beta (con $x,y\in\mathbb{R}^+$):
	\begin{equation*}
		\mathrm{B}(x,y)=\frac{\Gamma(x)\Gamma(y)}{\Gamma(x+y)}
	\end{equation*}
\end{theorem}
\begin{proof}
	Fissato $y$ reale positivo, sia $f_y:\mathbb{R^+}\to\mathbb{R^+}$ la funzione che rispetta:
	\begin{equation}\label{QuasiGamma}
		f_y(x)=\frac{\mathrm{B}(x,y)\Gamma(x+y)}{\Gamma(y)}
	\end{equation}
	
	Dimostro che $f_y$ rispetta le ipotesi di \cref{BohrMollerup}.
	\begin{itemize}
		\item $f_y(1)=1$
		
			Vale, sfruttando \cref{QuasiGamma,FunzionaleGamma} che:
			\begin{equation}\label{QuasiGammaDiUno}
				f_y(1)=\frac{\mathrm{B}(1,y)\Gamma(1+y)}{\Gamma(y)}=\mathrm{B}(1,y)y
			\end{equation}
			
			Però dalla definizione \cref{FunzioneBeta} e dalla simmetria \cref{BetaSimmetrica} ho anche:
			\begin{equation}\label{BetaDiUno}
				\mathrm{B}(1,y)=\mathrm{B}(y,1)=\int_0^1 t^{y-1} \mathrm{d}t = \left[\frac{t^y}y\right]_0^1= \frac1y
			\end{equation}
			
			Sostituendo \cref{BetaDiUno} nella \cref{QuasiGammaDiUno} ottengo $f_y(1)=1$.
		\item $f_y$ è log-convessa
		
			Applicando la log convessità di Gamma e Beta (\cref{BetaLogConvessa,GammaLogConvessa}) 
			nella \cref{QuasiGamma} ho che $f_y$ è prodotto di funzioni log-convesse, perciò è essa stessa log-convessa.
		\item $f_y(x+1)=xf_y(x)$
		
			Applicando l'equazione funzionale della Gamma nella \cref{QuasiGamma} si ha facilmente:
			\begin{equation}\label{QuasiFunzionaleBeta}
				f_y(x+1)=\frac{\mathrm{B}(x+1,y)(x+y)\Gamma(x+y)}{\Gamma(y)}=(x+y) f_y(x) \frac{B(x+1,y)}{\mathrm{B}(x,y)}
			\end{equation}
			Ed ora applico \cref{FunzionaleBeta3} nella \cref{QuasiFunzionaleBeta} ottenendo quando desiderato:
			\begin{equation*}
				f_y(x+1)=(x+y) f_y(x) \frac{x}{x+y}=xf_y(x)
			\end{equation*}
	\end{itemize}
	Poichè $f_y$ rispetta tutte le ipotesi di \cref{BohrMollerup}, lo applico ottenendo che $f_y=\Gamma$.
	
	Di conseguenza, ricordando la definizione \cref{QuasiGamma} ottengo:
	\begin{equation*}
		\frac{\mathrm{B}(x,y)\Gamma(x+y)}{\Gamma(y)}=f_y(x)=\Gamma(x) \Longrightarrow \mathrm{B}(x,y)=\frac{\Gamma(x)\Gamma(y)}{\Gamma(x+y)}
	\end{equation*}
\end{proof}

\begin{remark}[Valore di $\Gamma\left(\frac12\right)$]
	Ponendo $x=y=1$ in \cref{GammaBeta} e sfruttando \cref{BetaTrigonometrica} ottengo:
	\begin{equation*}
		\mathrm{B}\left(\frac12,\frac12\right)=\dfrac{\Gamma\left(\frac12\right)^2}{\Gamma(1)}\Rightarrow 
		\Gamma\left(\frac12\right)^2=2\int_0^{\frac{\pi}2}\sin^0u\cos^0u\mathrm{d}u=\pi\Rightarrow \Gamma\left(\frac12\right)=\sqrt{\pi}
	\end{equation*}
\end{remark}
\begin{theorem}[Formula di riflessione]
\label{Riflessione}
Per $0<x<1$ reale vale:
\begin{equation*}
	\Gamma(x)\Gamma(1-x)=\frac{\pi}{\sin(\pi x)}
\end{equation*}

\end{theorem}

\begin{proof}
Uso l'identità \ref{GaussFormula} per ottenere:
\begin{equation}
\label{MezzaRiflessione}
\begin{split}
\Gamma(x)\Gamma(1-x) & = \lim_{n\to\infty} \dfrac{n^xn!}{x(x+1)\cdots (x+n)} \cdot \dfrac{n^{1-x}n!}{(1-x)(1-x+1)\cdots (1-x+n)}\\
 & =\lim_{n\to\infty} \frac{n}{n+1-x} \cdot \frac{1}{x} \cdot \prod_{k=1}^{n}\dfrac{k^2}{k^2-x^2} \\
  & =\left(\lim_{n\to\infty} \frac{n}{n+1-x} \right) \cdot \left( \lim_{n\to\infty} x \prod_{k=1}^{n}\left(1-\frac{x^2}{k^2}\right) \right)^{-1}
\end{split}
\end{equation}

Ed ora ricordando la formula come prodotto infinito del seno:
\begin{equation*}
	\sin(\pi x)=\pi x \prod_{k=0} \left(1-\frac{x^2}{k^2}\right)
\end{equation*}

Ottengo che l'ultimo membro della \eqref{MezzaRiflessione} risulta $\frac{\pi}{\sin(\pi x)}$, che è la tesi del teorema.


\end{proof}

\begin{theorem}[Formula di duplicazione]
\label{Duplicazione}
Per $0<x$ reale vale:
\begin{equation*}
	\Gamma(x)\Gamma\left(x+\frac12\right)=2^{1-2x}\sqrt{\pi}\Gamma(2x)
\end{equation*}

\end{theorem}

\begin{proof}
Uso l'identità \ref{GaussFormula} per ottenere:
\begin{equation}
\label{MezzaDuplicazione}
\begin{split}
\Gamma(x)\Gamma\left(x+\frac12\right) = \lim_{n\to\infty} 
\dfrac{n^xn!}{x(x+1)\cdots (x+n)} \cdot \dfrac{n^{x+\frac12} n!}{\left(x+\frac12\right)\left(x+\frac32\right)\cdots \left(x+\frac{2n+1}2\right)}\\
 =\lim_{n\to\infty} \dfrac{(2n)^{2x}}{2^{2x}}\sqrt{n}\cdot (2n)!\cdot\binom{2n}{n}^{-1}\cdot \dfrac{2^{2n+1}}{(2x)(2x+1)\cdots(2x+2n)}\cdot \dfrac{1}{x+\frac12+n} \\
 =2^{1-2x} \left(\lim_{n\to\infty} \dfrac{ (2n)^{2x}(2n)! }{ (2x)(2x+1)\cdots(2x+2n) }\right) 
  \left( \lim_{n\to\infty} \frac{2^{2n}\sqrt{n}}{x+\frac12+n} \binom{2n}{n}^{-1}\right)
\end{split}
\end{equation}

Ricordando ancora la \ref{GaussFormula} si nota che il primo dei due limiti vale $\Gamma(2x)$.\\
Mentre per il secondo basta sfruttare la stima asintotica:
\begin{equation*}
	\binom{2n}{n}\sim \frac{4^n}{\sqrt{\pi n }}
\end{equation*}
per avere che è uguale a $\sqrt{\pi}$.\\
Riunendo i risultati ottenuti si ottiene la formula di duplicazione.

\end{proof}

\section{Differenziazione della gamma}\label{dg}
Questa sezione è la prima a dover usare esplicitamente ragionamenti del tipo \emph{epsilon, delta}.
Per evitare di usare questi ragionamenti si dovrebbero applicare teoremi forti riguardo la possibilità
di scambiare tra loro gli operatori di integrale, di derivata e di limite.
Tuttavia questi teoremi prescindono dal programma di analisi I e perciò abbiamo deciso di trovare strade
che li evitino, mantenendo le dimostrazioni le più elementari possibili.

Dimostreremo che la Gamma è una funzione derivabile infinite volte ed espliciteremo le sue derivate.
Inoltre studieremo alcune proprieta della digamma (derivata logartmica della gamma).

\begin{definition}
	Sia $f_h:\mathbb{R}\to\mathbb{R}$, con $h\in\mathbb{R^+}$, la funzione definita come:
	\begin{equation*}
		f_h(t)=\frac{t^h-1}h
	\end{equation*}

\end{definition}

\begin{lemma}\label{dg:LagrangeApprox}
	Fissati $t,h\in\mathbb{R^+}$, esiste $0\le h'\le h$ tale che:
	\begin{equation*}
		f_h(t)=\log{t}\cdot t^{h'}
	\end{equation*}
\end{lemma}
\begin{proof}
	Sia $f:\mathbb{R}\to\mathbb{R}$ la funzione definita come $g(x)=t^x$.
	
	Applicando le regole standard di derivazione si ha $g'(x)=\log{t}\cdot t^x$.
	
	Applico il teorema di lagrange con estremi $[0,h]$ alla funzione $g$ ottenendo la tesi:
	\begin{equation*}
		\exists 0\le h'\le h\ :\ f_h(t)=\frac{g(h)-g(0)}{h-0}=g'(h')=\log{t}\cdot t^{h'}
	\end{equation*}
\end{proof}



\begin{lemma}\label{dg:DisEstremale}
	Fissati $t,h\in\mathbb{R^+}$ con $h\le1$ risulta vera la disuguaglianza:
	\begin{equation*}
		\left\lvert f_h(t)-\log{t}\right\rvert\le \left\lvert\log{t}\right\rvert\max(1,t)
	\end{equation*}
\end{lemma}
\begin{proof}
	Applicando \cref{dg:LagrangeApprox} ottengo che vale la catena di identità (con $0<h'<h$):
	\begin{equation}\label{dg:FurbaId}
		\left\lvert f_h(t)-\log{t}\right\rvert=\left\lvert\log{t}\right\rvert \cdot \left\lvert t^{h'}-1\right\rvert
	\end{equation}
	
	Per $t\ge 1$ ho che vale (sfruttando $h'\le h\le 1$) $0\le t^{h'}-1\le t-1<t$.\\
	Per $0\le t<1$ ho che vale (sfruttando $h'\ge0$) $-1\le t^{h'}-1\le 0$.\\
	Unendo questi due risultati ottengo facilmente:
	\begin{equation}\label{dg:StupidaDis}
		\left\lvert t^{h'}-1\right\rvert \le \max(1,t)
	\end{equation}
	Applicando \cref{dg:StupidaDis} in \cref{dg:FurbaId} ottengo la tesi del lemma.
\end{proof}

\begin{lemma}\label{dg:UnifConv}
	Fissato un intervallo $[a,b]$ con $0<a<b$, per $h\to 0$ le funzioni $f_h$ convergono uniformemente alla funzione $\log$.
\end{lemma}
\begin{proof}
	Questa è una facile conseguenza di \cref{dg:LagrangeApprox}.
\end{proof}

\begin{theorem}\label{dg:GammaDerivata}
	La funzione Gamma è derivabile e la derivata rispetta:
	\begin{equation*}
		\Gamma'(x)=\int_0^{\infty} \log{t}\cdot t^{x-1}e^{-t}\mathrm{d}t
	\end{equation*}
\end{theorem}
\begin{proof}
	Come per la dimostrazione di \cref{GammaConverge}, l'integrale, che devo dimostrare
	essere la derivata della Gamma, esiste finito.
	
	Assunto che l'integrale esiste, la tesi del teorema è equivalente (per definizione di derivata) a dimostrare
	che per ogni $\varepsilon>0$ esiste $\delta$ (che scelgo minore di 1) tale che $\forall\ 0<h<\delta$ risulta:
	\begin{equation}\label{dg:EpsDeltaDerivata}
		\left\lvert 
		\frac{\Gamma(x+h)-\Gamma(x)}{h}-
		\int_0^{\infty} \log{t}\cdot t^{x-1}e^{-t}\mathrm{d}t
		\right\rvert \le \varepsilon
		\Longleftrightarrow
		\left\lvert
		\int_0^{\infty} \left(f_h(t)-\log{t}\right)t^{x-1}e^{-t}\mathrm{d}t
		\right\rvert \le \varepsilon
	\end{equation}
	
	Sia $0<a<1$ tale che:
	\begin{equation}\label{dg:ApproxIn0}
		\int_0^a \left\lvert \log(t)t^{x-1}e^{-t} \right\rvert \mathrm{d}t \le \frac{\varepsilon}3
	\end{equation}
	Tale $a$ esiste poichè l'integrale della \cref{dg:ApproxIn0} in 0 converge.
	
	Sia $b>\max(a,1)$ tale che:
	\begin{equation}\label{dg:ApproxInInf}
		\int_b^{\infty} \left\lvert (\log{t}\cdot t)t^{x-1}e^{-t}\right\rvert \mathrm{d}t \le \frac{\varepsilon}3
	\end{equation}
	Analogamente a prima, la $b$ esiste poichè l'integrale della \cref{dg:ApproxInInf} converge.
	
	Applicando i risultati \cref{dg:DisEstremale,dg:ApproxIn0,dg:ApproxInInf} ottengo le seguenti due identità:
	\begin{equation}\label{dg:IntIn0}
		\left\lvert \int_0^a \left(f_h(t)-\log{t}\right)t^{x-1}e^{-t}\mathrm{d}t\right\rvert
		\le
		\int_0^a \left\lvert f_h(t)-\log{t} \right\rvert t^{x-1}e^{-t}\mathrm{d}t
		\le
		\int_0^a \left\lvert \log{t} \right\rvert t^{x-1}e^{-t}\mathrm{d}t
		\le
		\frac{\varepsilon}3
	\end{equation}
	\begin{equation}\label{dg:IntInInf}
		\left\lvert \int_b^{\infty} \left(f_h(t)-\log{t}\right)t^{x-1}e^{-t}\mathrm{d}t\right\rvert
		\le
		\int_b^{\infty} \left\lvert f_h(t)-\log{t} \right\rvert t^{x-1}e^{-t}\mathrm{d}t
		\le
		\int_b^{\infty} \left\lvert \log{t}\cdot t\right\rvert t^{x-1}e^{-t}\mathrm{d}t
		\le
		\frac{\varepsilon}3
	\end{equation}
	
	Ora considero le funzioni $f_h$ ridotte all'intervallo $[a,b]$.
	L'uniforme convergenza mostrata in \cref{dg:UnifConv} e la limitatezza su $[a,b]$ di $t^{x-1}e^{-t}$ mi assicurano
	l'esistenza di $\delta>0$ tale che:
	\begin{equation}\label{dg:IntMiddle}
		\left\lvert \int_a^b \left(f_h(t)-\log{t}\right)t^{x-1}e^{-t}\mathrm{d}t\right\rvert \le \frac{\varepsilon}3
	\end{equation}
	Questo $\delta$ esiste poichè l'integrale converge a 0 per $h\to 0$.
	
	Unendo i risultati \cref{dg:IntIn0,dg:IntInInf,dg:IntMiddle}, con $h\le\delta$ ottengo
	proprio la \cref{dg:EpsDeltaDerivata} e quindi la tesi:
	\begin{multline}
		\left\lvert
		\int_0^{\infty} \left(f_h(t)-\log{t}\right)t^{x-1}e^{-t}\mathrm{d}t
		\right\rvert \le \\
		\left\lvert \int_0^a \left(f_h(t)-\log{t}\right)t^{x-1}e^{-t}\mathrm{d}t\right\rvert +
		\left\lvert \int_a^b \left(f_h(t)-\log{t}\right)t^{x-1}e^{-t}\mathrm{d}t\right\rvert +
		\left\lvert \int_b^{\infty} \left(f_h(t)-\log{t}\right)t^{x-1}e^{-t}\mathrm{d}t\right\rvert \\
		\le \varepsilon
	\end{multline}
\end{proof}
\begin{corollary}\label{dg:GammaDerivataN}
	La derivata $n$-esima della Gamma rispetta:
	\begin{equation*}
		\Gamma^{(n)}(x)=\int_0^{\infty} \log^{n}{t}\cdot t^{x-1}e^{-t}\mathrm{d}t
	\end{equation*}
\end{corollary}
\begin{proof}
	Si dimostra agevolmente per induzione su $n$. In particolare il passo induttivo si svolge
	ripetendo pedissequamente la dimostrazione di \cref{dg:GammaDerivataN}, solo sostituendo ovunque $t^{x-1}e^{-x}$
	con $\log^{n-1}{t}\cdot t^{x-1}e^{-x}$.
\end{proof}





\section{Approssimazione della funzione Gamma}
Concludiamo queste dispense proponendo due dimostrazioni intrinsecamente distinte dell'approssimazione di Stirling per
la funzione Gamma. 

È importante notare che la prima dimostrazione è interamente autocontenuta (a parte il primo risultato, che è un fatto
noto di analisi) mentre la seconda usa pesantemente vari fatti dimostrati nelle sezioni precedenti.

\begin{lemma}\label{ga:LimiteIntegrali}
	Siano $f_n$, con $n\in\mathbb{N}$, e $g$ funzioni definite in $(0,\infty)$ e Riemann-integrabili su $[a,b]$ per ogni
	$0<a<b<\infty$. Se valgono le seguenti proprietà:
	\begin{itemize}
		\item $|f_n|\le g$ per ogni $n\in\mathbb{N}$;
		\item $f_n\to f$ uniformemente in ogni intervallo chiuso di $(0,\infty)$;
		\item $\int_0^\infty{g(x)dx}<\infty$.
	\end{itemize}
	Allora vale che:
	\begin{equation*}
		\lim_{n\to\infty}\int_0^\infty f_n(x)dx=\int_0^\infty f(x)dx
	\end{equation*}
\end{lemma}

\begin{proof}
	Dimostro innanzitutto che per ogni $0<a<b<\infty$, ho che $f$ è integrabile su $[a,b]$ e in particolare vale:
	\begin{equation*}
		\int_a^b{f(x)dx}=\lim_{n->\infty}\int_a^b{f_n(x)dx}
	\end{equation*}
	Sia $\sigma_m$ la suddivisione equispaziata dell'intervallo $[a,b]$ di nodi $x_i=a+\frac{i}{m}(b-a)$,
	allora $f$ è Riemann-integrabile su $[a,b]$ se per ogni $\varepsilon>0$ esiste $M$ tale che se $m\ge M$
	allora $|S(f,\sigma_m)-s(f,\sigma_m)|<\varepsilon$.
	Per la disuguaglianza triangolare vale però che:
	\begin{equation*}
		|S(f,\sigma_m)-s(f,\sigma_m)|\le  
		|S(f,\sigma_m)-S(f_n,\sigma_m)|+ |s(f_n,\sigma_m)-s(f,\sigma_m)|+ |S(f_n,\sigma_m)-s(f_n,\sigma_m)|
	\end{equation*}
	Dato che $f_n\to f$ uniformemente su $[a,b]$, allora per ogni $\mu>0$ esiste $N$ tale che per ogni
	$n\ge N$ e per ogni $x\in[a,b]$ vale $|f(x)-f_n(x)|<\mu$, quindi vale facilmente che per ogni $m\ge M$:
	\begin{equation} \label{ViciniInIntervallo}
	\begin{split}
		|S(f,\sigma_m)-S(f_n,\sigma_m)| & <\mu(b-a) \\
		|s(f,\sigma_m)-s(f_n,\sigma_m)| & <\mu(b-a)
	\end{split}
	\end{equation} 
	E dato che $f_n$ è Riemann-integrabile
	su $[a,b]$ per ogni $n$, allora per ogni $\mu>0$ esiste $M$ tale che per ogni $m\ge M$ vale:
	\begin{equation}
		|S(f_n,\sigma_m)-s(f_n,\sigma_m)|<\mu
	\end{equation}
	Ma allora per ogni $\mu>0$ esistono $N$ e $M$ tali che per ogni $n\ge N$ e $m\ge M$ vale:
	\begin{gather*}
		|S(f,\sigma_m)-S(f_n,\sigma_m)|+ |s(f_n,\sigma_m)-s(f,\sigma_m)|+ |S(f_n,\sigma_m)-s(f_n,\sigma_m)|<\mu(2b-2a+1) \\
		\Longrightarrow |S(f,\sigma_m)-s(f,\sigma_m)|< \mu(2b-2a+1)
	\end{gather*}
	Quindi scegliendo $\mu=\varepsilon/(2b-2a+1)$ ottengo che per ogni $m\ge M$ vale:
	\begin{equation*}
	|S(f,\sigma_m)-s(f,\sigma_m)|< \varepsilon
	\end{equation*}
	Quindi $f$ è Riemann-integrabile su $[a,b]$, e in particolare dalla \cref{ViciniInIntervallo} si ottiene facilmente
	anche che
	\begin{equation}\label{IntegraleInIntervallo}
		\int_a^b{f(x)dx}=\lim_{n->\infty}\int_a^b{f_n(x)dx}
	\end{equation}
	Da quest'ultima relazione ottengo anche che $|f|$ è Riemann-integrabile in $[a,b]$, poichè in generale se
	$h$ è una funzione Riemann-integrabile in $[a,b]$ lo è anche $|h|$.\\
	Ora, dato che $|f_n|\le g$ per ogni $n\in\mathbb{N}$, passando al limite ottengo che $|f(x)|\le g(x)$ per ogni $x\in(0,\infty)$.
	Di conseguenza, dato che per quanto già detto $|f|$ è Riemann-integrabile in $[a,b]$, $|f|$ è Riemann-integrabile
	anche in $(0,\infty)$ perchè è non negativa.\\
	Ma dato che esiste l'integrale improprio di $|f|$ su $(0,\infty)$, allora esiste anche l'integrale improprio di $f$ su 
	$(0,\infty)$, e analogamente a quanto detto prima su un intervallo, vale proprio:
	\begin{equation*}
		\int_0^\infty{f(x)dx}=\lim_{n->\infty}\int_0^\infty{f_n(x)dx}
	\end{equation*}
\end{proof}

\begin{theorem}[Approssimazione di Stirling per la funzione Gamma]\label{StirlingGamma}
	La formula di Stirling offre un'approssimazione per $\Gamma(x+1)$:
	\begin{equation*}
		\lim_{x\to\infty}\frac{\Gamma(x+1)}{(x/e)^x\sqrt{2\pi x}}=1
	\end{equation*}
\end{theorem}
\begin{proof}
	Sostituendo $t=x(1+s\sqrt{2/x})$ nella definizione della funzione Gamma \cref{FunzioneGamma} ottengo:
	\begin{equation} \label{ga:QuasiStirling}
	\begin{split}
		\Gamma(x+1) & = \int_0^\infty{e^{-t}t^{x}dt}\\
					& = \int_{-\sqrt{\frac{x}{2}}}^\infty{ e^{-x(1+s\sqrt{2/x})} x^x\left(1+s\sqrt{\frac{2}{x}}\right)^x \sqrt{2x} ds}\\
					& = e^{-x}x^x\sqrt{2x}\int_{-\sqrt{\frac{x}{2}}}^\infty{ \left[e^{-s\sqrt{2/x}} \left(1+s\sqrt{\frac{2}{x}}\right)\right]^x ds}\\
					& = e^{-x}x^x\sqrt{2x}\int_{-\sqrt{\frac{x}{2}}}^\infty{ e^{-s\sqrt{2x}+x\ln\left(1+s\sqrt{\frac{2}{x}}\right)} ds}\\
					& = e^{-x}x^x\sqrt{2x}\int_{-\sqrt{\frac{x}{2}}}^\infty{ e^{-s^2\left(\frac{\sqrt{2x}}{s}-\frac{x}{s^2}\ln\left(1+s\sqrt{\frac{2}{x}}\right)\right)} ds}
	\end{split}
	\end{equation}
	Ora definisco per comodità $h_x(s)=\frac{\sqrt{2x}}{s}-\frac{x}{s^2}\ln\left(1+s\sqrt{\frac{2}{x}}\right)$ 
	e mi concentro su quest'ultimo integrale, in particolare esso è uguale a:
	\begin{equation*}
		\int_{-\infty}^\infty{ f_x(s) ds}
	\end{equation*}
	dove 
	\begin{equation*}
		f_x(s)=\begin{cases}
				e^{-s^2h_x(s)}, & \mbox{se } -\sqrt{x/2}<s<\infty \\
							 0, & \mbox{se } s\le -\sqrt{x/2}
	\end{cases}
	\end{equation*}
	
	Dimostro innanzitutto che $f_x(s)\to e^{-s^2}$ uniformemente su $[a,b]$ per $x\to\infty$, con 
	$-\infty<a<b<\infty$. Ciò equivale a dimostrare che $h_x(s)\to 1$ uniformemente su $[a,b]$ per $x\to\infty$.
	Sostituendo a $\ln\left(1+s\sqrt{\frac{2}{x}}\right)$ il suo sviluppo di Taylor ottengo la seguente identità:
	\begin{equation*}
	\begin{split}
		h_x(s)	& =\frac{\sqrt{2x}}{s}-\frac{x}{s^2}\ln\left(1+s\sqrt{\frac{2}{x}}\right)\\
				& =\frac{\sqrt{2x}}{s}-\frac{x}{s^2}\left(s\sqrt{\frac{2}{x}}-\frac{s^2}{x}+\bigO\left(\frac{s^3}{x\sqrt{x}}\right) \right)\\
				& =1+\bigO\left( \frac{s}{\sqrt{x}} \right)
	\end{split}
	\end{equation*}
	Quindi esistono $H,K$ tali che, se $\left\lvert\frac s{\sqrt x}\right\rvert<H$ allora vale 
	$|h_x(s)-1|<K\frac{|s|}{\sqrt{x}}$ per ogni $s\in [a,b]$.
	
	Allora in particolare, ponendo $M=\max_{s\in[a,b]}\{|s|\}$, risulta che, se $\left\lvert\frac M{\sqrt x}\right\rvert<H$,
	allora $|h_x(s)-1|<K\frac{M}{\sqrt{x}}$. Ma questo equivale a dire:
	\begin{equation*}
		h_x(s)=1+\bigO\left( \frac{1}{\sqrt{x}} \right)
	\end{equation*}
	Che implica facilmente $h_x(s)\to 1$ uniformemente su $[a,b]$ per $x\to\infty$.
	
	Ora distinguo due casi:
	\begin{itemize}
		\item	Se $s<0$ ho che $0\le f_x(s) \le e^{-s^2}$, utilizzando l'uniforme convergenza di $f_x(s)$ in ogni
				intervallo e \cref{GaussIntegral}, posso applicare \cref{ga:LimiteIntegrali} alle funzioni $f_x(s)$ 
				e ottengo:
				\begin{equation}\label{ga:PartialIntegralNeg}
					\lim_{x\to\infty} \int_{-\infty}^0{f_x(s)ds}=\int_{-\infty}^0{\left(\lim_{x\to\infty}f_x(s)\right)ds}
					=\int_{-\infty}^0{e^{-s^2}ds}=\frac{\sqrt{\pi}}{2}
				\end{equation}
		\item	Se $s>0$ ho che $0\le f_x(s) \le f_1(s)$, in quanto $h_x(s)\ge h_1(s)$. Inoltre, grazie
				alla \cref{FunzioneGamma}, vale che:
				\begin{equation*}
				\begin{split}
					\int_0^{\infty}{f_1(s)ds}	&=\int_0^\infty{e^{-\sqrt{2}s}\left(1+\sqrt{2}s \right)ds}\\
												&=\frac{e}{\sqrt{2}}\int_0^\infty{e^{-t}t dt}\\
												&=\frac{e}{\sqrt{2}}\Gamma(2)<\infty
				\end{split}
				\end{equation*}
				Quindi, utilizzando la convergenza uniforme di $f_x(s)$ a $e^{-s^2}$ in ogni intervallo
				$[a,b]$, posso nuovamente applicare \cref{ga:LimiteIntegrali}, da cui ottengo:
				\begin{equation}\label{ga:PartialIntegralPos}
					\lim_{x\to\infty} \int_0^\infty{f_x(s)ds}=\int_0^\infty{\left(\lim_{x\to\infty}f_x(s)\right)ds}
					=\int_0^\infty{e^{-s^2}ds}=\frac{\sqrt{\pi}}{2}
				\end{equation}
	\end{itemize}
	
	Infine, unendo \cref{ga:PartialIntegralNeg} e \cref{ga:PartialIntegralPos}, ottengo:
	\begin{equation*}
		\lim_{x\to\infty}\int_{-\infty}^\infty{ f_x(s) ds}=\sqrt{\pi}
	\end{equation*}
	da cui passando al limite nella \cref{ga:QuasiStirling} ottengo la tesi.
\end{proof}

\begin{lemma}\label{ga:ApproxReali}
	Vale la seguente stima
	\begin{equation*}
		\log{\Gamma(x+1)}=\log{\Gamma(\partint{x} +1)}+\{x\}\log(x)+\bigO\left(\frac 1x\right)
	\end{equation*}
\end{lemma}
\begin{proof}
	La dimostrazione segue le linee di quella di \cref{dg:RapportoAsint}.
	
	Applico il teorema di Lagrange sulla funzione $\log\Gamma$, sfruttando \cref{dg:psiApprox}, ottenendo che esiste
	$\partint{x}\le x'\le x$ tale che:
	\begin{equation}
		\log{\Gamma(x+1)}-\log{\Gamma(\partint{x} +1)}=\{x\}\psi(y)=\{x\}\log(y)+\bigO\left(\frac 1y\right)=
		\{x\}\log(x)+\bigO\left(\frac 1x\right)
	\end{equation}
	che è la tesi.
\end{proof}

\begin{lemma}\label{ga:StupidaApprox}
	Vale la seguente approssimazione:
	\begin{equation*}
		\partint{x}(\log(\partint{x})-1)=\partint{x} \log(x)-x+\smallO(1)
	\end{equation*}
\end{lemma}
\begin{proof}
	Applicando Lagrange (e implicitamente delle approssimazioni sull'inverso):
	\begin{equation}\label{ga:NoiaApprox}
		\log{x}=\log{\partint{x}} + \frac{\{x\}}{\partint{x}} +\smallO\left(\frac 1{\partint{x}}\right)
	\end{equation}
	
	Applicando \cref{ga:NoiaApprox} vale la seguente catena di uguaglianze:
	\begin{equation*}
		\partint{x}(\log(\partint{x})-1)=\partint{x}\log{x}-\{x\}+
		\partint{x}\smallO\left(\frac 1{\partint{x}}\right)-\partint{x}=
		\partint{x}\log{x}-x+\smallO(1)
	\end{equation*}
	e questo è equivalente alla tesi.
\end{proof}



\begin{proof}[Seconda dimostrazione di Stirling]
	Applicando \cref{ga:ApproxReali,ga:StupidaApprox,f:StirlingNaturali} ho:
	\begin{equation*}\begin{split}
		\log{\Gamma(x+1)}&=\log\Gamma(\partint{x}+1)+\{x\}\log{x}+\bigO\left(\frac 1x\right)\\
		&=\partint{x}\left(\log{x}-1\right)+\log{\sqrt{2\pi\partint{x}}}+\{x\}\log{x}+\bigO\left(\frac 1x\right)\\
		&=\partint{x} \log(x)-x+\smallO(1)+\log{\sqrt{2\pi x}}+\{x\}\log{x}+\bigO\left(\frac 1x\right)\\
		&=x(\log{x}-1)+\log{\sqrt{2\pi x}}+\smallO(1)
	\end{split}\end{equation*}
	Che è proprio l'approssimazione di Stirling.
\end{proof}








\end{document}

\makeindex